\chapter{Wstęp}
Wzrost potencjału współczesnej nauki jest możliwy dzięki cały czas rozwijającemu się zapleczu technicznemu. To właśnie ono sprawia, że człowiek może widzieć więcej, słyszeć to co do tej pory było niesłyszalne, doświadczać tego co było nieodkryte. Niewątpliwie w~rozwój nauk przyrodniczych, medycyny, fizyki i~wielu innych miały bezprzewodowe sieci czujników. Pozwoliły one na całodobowy pomiar danej, charakterystycznej dla siebie wielkości.

Dzięki postępom techniki w~zakresie energooszczędnych procesorów, komunikacji oraz scalonych układów elektronicznych możliwe stało się stworzenie małych, tanich oraz mających niski pobór energii urządzeń zawierających w~sobie dowolny zestaw sensorów oraz aktuatorów. Urządzenia te można łączyć w~wielowięzłowe sieci, co zapoczątkowało rozwój dziedziny zajmującej się bezprzewodowymi sieciami czujnikowymi \cite{FahmyIntro2016}. 

Sieć bezprzewodową charakteryzuje brak sterowania centralnego, który umożliwia rozproszenie dużej liczba czujników w~danym, interesującym badacza obszarze. Sieci te same w~sobie zapewniają element losowości, który jest bardzo często pożądany w~badaniach, ponieważ nie jest możliwe projektowanie dokładnego rozmieszczenia danego czujnika. Najistotniejszą funkcją sieci jest pozyskanie informacji z~obszaru, w~którym znajdują się sensory \cite{Akyildiz2010}.

Bezprzewodowe sieci czujników charakteryzuje szeroki wachlarz zastosowań. Można tu wyróżnić wspomniane wcześniej cele naukowe, a~w szczególności te dotyczące badania świata przyrody. Na uwagę zasługują również te sieci, które zostały zamontowane w~celach ochronnych takie jak miejski monitoring, czy prewencja włamań do domu, a~także systemy ostrzegania o~pożarze. Systemy te wykorzystywane są również w~miejscach pracy tworząc inteligentne przedszkola czy winnice. Można więc śmiało powiedzieć, że
stały się one częścią naszego życia \cite{Ilyas2004}.

Ponadto bezprzewodowe sieci czujnikowe wykorzystywane są z~powodzeniem w~prężnie rozwijających się technologiach IoT (Internet of Things). 

Celem pracy było przeprowadzenie badań protokołów trasowania WSN, aby zrealizować to zadanie konieczne było przygotowanie zestawu symulacji wybranych algorytmów oraz przetestowanie ich za pomocą scenariuszy testowych i~zebranie wyników. Otrzymane wyniki zostały przetworzone, zwizualizowane oraz została przeprowadzona ich analiza.

Z racji wykorzystywanej w~publikacjach terminologii nie mającej ugruntowanych polskojęzycznych odpowiedników zostały w~pracy zastosowane następujące tłumaczenia:
\begin{itemize}
\item Cluster Head (CH) - lokalny węzeł bazowy
\item Sink - stacja bazowa
\item Transceiver - nadajnikoodbiornik
\end{itemize}
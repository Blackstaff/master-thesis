\chapter{Wnioski}

Ogólna tendencja - im większy rozmiar pakietu oraz krótszy odstęp, tym  szybsze tempo wyłączania się węzłów sieci.

Algorytmy Flood oraz SPIN wykazują jednakowe tempo spadku liczby aktywnych czujników, niezależne od rozmiaru pakietu oraz okresu między pakietami. W przypadku również większość czujników wyłącza się w bardzo krótkim przedziale czasowym. Oznacza to, że zużycie energii przez węzły jest równomierne (wszystkie czujniki są przez cały czas swojej aktywności w trybie nasłuchiwania - w przeciwieństwie do rodziny alogrytmów typu LEACH, w których występują okresy ''uśpienia'' czujników).

Sieci korzystające z protokołów LEACH, ALEACH i LEACH DCHS wykazują wyraźnie dłuższe działanie niż sieci używające Flood i SPIN.
Różnice zacierają się dla większych rozmiarów pakietów i mniejszych odstępów czasu pomiędzy pakietami.

Różnice pomiędzy grupami protokołów są wyraźniejsze dla sieci z liczbą węzłów 200.

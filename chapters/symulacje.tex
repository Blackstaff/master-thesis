\chapter{Symulacje}
\section{Konfiguracja}
W ramach pracy został przeprowadzony zestaw scenariuszy symulacyjnych. Każdy z zaimplementowanych protokołów został przetestowany z przygotowanymi zestawami parametrów wejściowych.
Do tych parametrów należą:
\paragraph{Dystrybucja węzłów sieci}
Symulacje zostały przeprowadzone dla dwóch sposobów dystrybucji węzłów: zgodnej z rozkładem normalnym oraz zgodnej z rozkładem jednorodnym.

Dla rozkładu normalnego średnia współrzędnej wynosiła 400m, a jej odchylenie standardowe 100m.

\begin{figure}[H]
	\begin{center}
		\includegraphics[scale=0.55]{\ImgPath/charts/normal_distribution.png}
	\end{center}
	\caption{Przykładowa dystrybucja węzłów sieci zgodna z rozkładem normalnym}
\end{figure}

Z rozkładem jednorodnym węzły zostały rozłożone na obszarze o kształcie prostokąta i wymiarach 450m na 550m.

\begin{figure}[H]
	\begin{center}
		\includegraphics[scale=0.55]{\ImgPath/charts/uniform_distribution.png}
	\end{center}
	\caption{Przykładowa dystrybucja węzłów sieci zgodna z rozkładem jednorodnym}
\end{figure}

\paragraph{Liczba węzłów sieci}
Symulacje przeprowadzono dla sieci składającej się z dwudziestu oraz dwustu czujników.
\paragraph{Rozmiar pakietu z danymi}
Symulacje przeprowadzono dla rozmiarów pakietów 5B, 50B, 500B, 5000B.
\paragraph{Częstotliwość generowania nowych danych}
Symulacje przeprowadzono dla okresów: 5s, 10s, 15s, 20s.
\paragraph{Początkowa wartość energii elektrycznej w węźle}
Jako początkową wartość dla czujnika przyjęto 5J.
%uzasadnić na krzywej zużycia baterii

Dodatkowo dla protokołów z rodziny LEACH zdefiniowane zostały parametry:
\paragraph{Procent cluster headów w sieci}
\paragraph{Długość rundy}
%\paragraph{Współczynnik kompresji danych przez cluster heady}

W celu zniwelowania wpływu losowości, dla każdego pojedynczego zestawu parametrów wykonano 5 przebiegów symulacji z różnym ziarnem. Wyniki tych przebiegów zostały uśrednione.

Na poniższym listingu znajduje się fragment pliku konfiguracyjnego symulacji.
\begin{verbatim}
**.numSensors = ${numSensors = 20, 200, 2000}
**.app.protocol = 59
**.app.packetLength = ${packetLength = 5, 50, 500, 5000}B
**.app.startTime = 1s + exponential(10ms)
**.networkLayerType = "FloodNetworkLayer"
**.sink.protocol = 59
**.broadcastSinkAddress = false
**.sink.nic.mac.address = "FF-FF-FF-FF-FF-FE"
**.app.destAddresses = "FF-FF-FF-FF-FF-FE"
**.np.sinkAddr = "FF-FF-FF-FF-FF-FE"
**.radio.displayCommunicationRange = true
**.app.sendInterval = ${sendInterval = 5, 10, 15, 20}s + exponential(10ms)

**.sensor[*].energyStorage.nominalCapacity = ${nodeEnergy = 5}J
**.sink.energyStorage.nominalCapacity = 4J * ${nodeEnergy}
**.energyStorage.nodeShutdownCapacity = ${nodeEnergy} * 0.1J
**.energyStorage.nodeStartCapacity = ${nodeEnergy} * 0.2J
\end{verbatim}
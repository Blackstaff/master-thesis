\chapter{Narzędzia do symulacji sieci}
% Wykorzystany symulator zdarzeń dyskretnych, opisać po krótce inne
% 1-2 strony
Rozdział ten opisuje narzędzia oraz technologie, które zostały użyte do symulacji działania sieci czujników, implementacji protokołów trasowania oraz przeprowadzenia testów.
\section{Przegląd narzędzi}
\cite{Xian2008}
\cite{Nayyar2015}
\subsection{NS-2}
\subsection{NS-3}
\subsection{IKR}
\subsection{openWNS}
\subsection{Matlab}
\subsection{OPNET}
\subsection{J-Sim}
\subsection{SensorSim}
\subsection{NCTUns}
\subsection{SSFNet}
\subsection{QualNet}
\subsection{SENSE}
\section{\omnetpp}
\omnetpp jest zbudowanym w sposób modularny symulatorem zdarzeń dyskretnych. Stanowi on ogólne narzędzie umożliwiające przeprowadzanie symulacji między innymi: przewodowych oraz bezprzewodowych sieci komputerowych, systemów wieloprocesorowych, chmur obliczeniowych czy też ruchu miejskiego. Dla każdej bardziej wyspecjalizowanej dziedziny konieczne jest stworzenie nowego modelu (zbioru modułów) lub skorzystanie z jednego z już istniejących rozwiązań (np. INET, VEINS).\cite{Varga2017}
\subsection{Moduły}
Podstawowym budulcem symulacji w \omnetpp są moduły. Dzielą się one na proste oraz złożone.
% Opis co to moduł oraz że są proste i złożone, możne je zagneżdżać oraz wiązać za pomocą bram (ang. gate). A komunikują się za pomocą wiadomości i sygnałów

Do implementacji modułów wykorzystuje się dwa języki
\begin{enumerate}
	\item NED - język domenowy \omnetpp, za pomocą którego definiowane są moduły. Zawierają opis parametrów oraz bram modułu. Dodatkowo w przypadku modułów złożonych definiowane są zagnieżdżone moduły wraz z ich połączeniami.
	%Wstawić przykład
	\item C++ - wykorzystywany do implementacji modułów prostych (moduły złożone definiowane są tylko za pomocą plików NED).
	%Wstawić przykład
\end{enumerate}
Dodatkowo, w celu przyspieszenia oraz ułatwienia implementacji wiadomości, które moduły wykorzystują w komunikacji stworzono odpowiedni język dziedzinowy, który tłumaczony jest przez kompilator do kodu C++.
%Wstawić przykład

Moduły umieszczane są w Sieciach (Network), które również zdefiniowane są w języku NED.

Do komunikacji moduły wykorzystują zdefiniowane (w pliku .ned) przez programistę łącza (ang. links ) oraz bramy (ang. gates).
\subsection{Przeprowadzenie symulacji}
\omnetpp umożliwia skorzystanie z różnych interfejsów użytkownika: QT, TKenv oraz wiersza poleceń. W fazie testowania poprawności przygotowanej symulacji skorzystać można z jednego z interfejsów graficznych. Po weryfikacji zalecana jest rezygnacja z interfejsu graficznego w celu przyspieszenia działania symulacji. Niebywałym ułatwieniem w przygotowaniu oraz uruchomieniu zestawu symulacji w celu zebrania danych do analizy są pliki konfiguracyjne. Umożliwiają one określenie parametrów modułów wchodzących w skład sieci oraz zadeklarowanie parametrów wchodzących w skład zmiennych badanych w ramach symulacji (parameter studies).
%Wstawić przykład
\section{INET}
INET jest biblioteką/modelem zawierającym implementacji wielu protokołów sieciowych (m.in. IPv4, IPv6, TCP, SCTP, UDP), jak również standardów komunikacji wykorzystywanych m.in. w sieciach czujników (IEEE 802.15.4).\cite{inet}

%Opisać, że biblioteka zawierała błędy, opisać jakie błędy i jak zostały naprawione
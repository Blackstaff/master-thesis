\chapter{Narzędzia do symulacji sieci}
% Wykorzystany symulator zdarzeń dyskretnych, opisać po krótce inne
% 1-2 strony
Rozdział ten opisuje narzędzia oraz technologie, które zostały użyte do symulacji działania sieci czujników, implementacji protokołów trasowania oraz przeprowadzenia testów.
\section{Przegląd narzędzi}
\cite{Xian2008}
\cite{Nayyar2015}
\subsection{Narzędzia komercyjne}
\paragraph{QualNet}
Komercyjny symulator sieci stworzony przez firmę Scalable Technologies. Umożliwia tworzenie modeli sieci składających się z trasowników, przełączników, serwerów, punktów dostępowych, radioodbiorników, anten wraz z gotowymi protokołami trasowania.

\paragraph{OPNET}
\paragraph{Matlab}
\paragraph{LabView}
\subsection{Narzędzia otwarte}
\paragraph{NS-2} Jest to uznany symulator zdarzeń dyskretnych stworzony w 1989 roku. W symulatorze wykorzystywane są dwa języki: C++ oraz OTcl (Object Oriented Tool Command Language). C++ używany jest do implementacji protokołów oraz rozszerzania biblioteki symulacyjnej. Z kolei w języku OTcl pisane są skrypty, których zadaniem jest konfiguracja symulacji, topologii sieci, tworzenie scenariuszy testowych i prezentacja wyników. Do jego zalet z punktu widzenia symulacji bezprzewodowych sieci czujnikowych należą gotowe implementacje protokołów takich jak np. 802.11, 802.16, 802.15.4. Niestety nie zapewnia on symulacji  procesów badanych przez czujniki oraz wykorzystywany przez niego model zużycia energii, formaty pakietów i protokoły MAC odbiegają znacznie od tych używanych w rzeczywistych implementacjach bezprzewodowych sieci czujnikowych. Brak jest również wsparcia dla warstwy aplikacji. NS-2 dostępny jest za darmo.
\paragraph{NS-3}
Podobnie jak NS-2, NS-3 jest również uznanym symulatorem zdarzeń dyskretnych. Nie stanowi on jednak rozszerzenia ani kontynuacji NS-3. Jest to kompletnie nowy symulator z całkowicie innym interfejsem programistycznym. W NS-3 wszystkie symulacje pisane są w czystym C++, z możliwością skorzystania z Pythona. Nie posiada on interfejsu graficznego - w celu graficznej prezentacji wyników konieczne jest skorzystanie z zewnętrznych narzędzi. Zawiera moduły implementujące 802.15.4, 6LoWPAN i RPL.
\paragraph{J-Sim}
Jest to symulator oparty o architekturę komponentową. Umożliwia korzystanie z różnych języków skryptowych, takich jak Perl, Tcl i Python. J-Sim zawiera bibliotekę dedykowaną do symulacji bezprzewodowych sieci czujnikowych. Umożliwia przeprowadzanie symulacji z liczbą węzłów przekraczającą 1000.
\paragraph{NetTopo Simulator}
%\subsection{IKR}
%\subsection{openWNS}
%\subsection{Matlab}
%\subsection{OPNET}
%\subsection{SensorSim}
%\subsection{NCTUns}
%\subsection{SSFNet}
%\subsection{QualNet}
%\subsection{SENSE}
\section{\omnetpp}
\omnetpp jest zbudowanym w sposób modularny symulatorem zdarzeń dyskretnych. Stanowi on ogólne narzędzie umożliwiające przeprowadzanie symulacji między innymi: przewodowych oraz bezprzewodowych sieci komputerowych, systemów wieloprocesorowych, chmur obliczeniowych czy też ruchu miejskiego. Dla każdej bardziej wyspecjalizowanej dziedziny konieczne jest stworzenie nowego modelu (zbioru modułów) lub skorzystanie z jednego z już istniejących rozwiązań (np. INET, VEINS).\cite{Varga2017}
\subsection{Moduły}
Podstawowym budulcem symulacji w \omnetpp są moduły. Dzielą się one na proste oraz złożone.
% Opis co to moduł oraz że są proste i złożone, możne je zagneżdżać oraz wiązać za pomocą bram (ang. gate). A komunikują się za pomocą wiadomości i sygnałów

Do implementacji modułów wykorzystuje się dwa języki
\begin{enumerate}
	\item NED - język domenowy \omnetpp, za pomocą którego definiowane są moduły. Zawierają opis parametrów oraz bram modułu. Dodatkowo w przypadku modułów złożonych definiowane są zagnieżdżone moduły wraz z ich połączeniami.
	%Wstawić przykład
	\item C++ - wykorzystywany do implementacji modułów prostych (moduły złożone definiowane są tylko za pomocą plików NED).
	%Wstawić przykład
\end{enumerate}
Dodatkowo, w celu przyspieszenia oraz ułatwienia implementacji wiadomości, które moduły wykorzystują w komunikacji stworzono odpowiedni język dziedzinowy, który tłumaczony jest przez kompilator do kodu C++.
%Wstawić przykład

Moduły umieszczane są w Sieciach (Network), które również zdefiniowane są w języku NED.

Do komunikacji moduły wykorzystują zdefiniowane (w pliku .ned) przez programistę łącza (ang. links ) oraz bramy (ang. gates).
\subsection{Przeprowadzenie symulacji}
\omnetpp umożliwia skorzystanie z różnych interfejsów użytkownika: QT, TKenv oraz wiersza poleceń. W fazie testowania poprawności przygotowanej symulacji skorzystać można z jednego z interfejsów graficznych. Po weryfikacji zalecana jest rezygnacja z interfejsu graficznego w celu przyspieszenia działania symulacji. Niebywałym ułatwieniem w przygotowaniu oraz uruchomieniu zestawu symulacji w celu zebrania danych do analizy są pliki konfiguracyjne. Umożliwiają one określenie parametrów modułów wchodzących w skład sieci oraz zadeklarowanie parametrów wchodzących w skład zmiennych badanych w ramach symulacji (parameter studies).
%Wstawić przykład
\section{INET}
INET jest biblioteką/modelem zawierającym implementacji wielu protokołów sieciowych (m.in. IPv4, IPv6, TCP, SCTP, UDP), jak również standardów komunikacji wykorzystywanych m.in. w sieciach czujników (IEEE 802.15.4).\cite{inet}

\newpage
\begin{center}
\large \bf
Symulacje wybranych algorytmów trasowania dla bezprzewodowych sieci czujnikowych
\end{center}

\section*{Streszczenie}
Niniejsza praca poświęcona jest symulacjom wybranych algorytmów trasowania w bezprzewodowych sieciach czujnikowych. Składa się ona z sześciu rozdziałów. Pierwszy z nich stanowi wstęp. Zawarte zostały w nim ogólne informacje dotyczące bezprzewodowych sieci czujnikowych oraz cel pracy. Kolejny z nich podejmuje tematykę bezprzewodowych sieci czujnikowych w sposób bardziej szczegółowy. Przedstawiona została definicja sieci czujnikowej, jej budowa oraz zastosowania. Znalazły się tu również opisy wybranych protokołów trasowania, które zostały zaimplementowane oraz zbadane w ramach pracy. Są to: Flood, SPIN, LEACH, ALEACH i LEACH-DCHS. Trzeci rozdział przedstawia motywację stojącą za wykorzystaniem oprogramowania do symulacji sieci oraz przedstawia przegląd komercyjnych oraz otwartych rozwiązań w tej dziedzinie. Szczególna uwaga poświęcona została oprogramowaniu \omnetpp, które zostało wykorzystane podczas tworzenia niniejszej pracy. W rozdziale pod tytułem ,,Implementacja'' opisane zostały szczegóły implementacyjne poszczególnych protokołów opisanych w rozdziale drugim. W ramach pracy powstała również biblioteka modułów \omenetpp, ułatwiająca budowanie symulacji bezprzewodowych sieci czujnikowych w tymże środowisku. Dodatkowo opisane zostały poprawki błędów w bibliotece INET bez których nie byłoby możliwe przeprowadzenie symulacji. Rozdział piąty opisuje przeprowadzone symulacje. Przedstawione zostały parametry wejściowe oraz sposób definicji scenariuszy symulacyjnych. Symulacje przeprowadzone zostały dla różnych zmiennych, którymi były: liczba węzłów sieci (czujników), rozmiar danych przesyłanych w pakiecie z danymi, czas pomiędzy kolejnymi pakietami z danymi. Następnie opisany został proces analizy oraz wizualizacji danych. W celu przetworzenia oraz poprawnego sformatowania danych uzyskanych po przeprowadzeniu symulacji wykorzystany został dedykowany, autorski skrypt napisany w języku Ruby. Wizualizacja danych została przeprowadzona z wykorzystanie języka R oraz pakietu ggplot2. W kolejnych akapitach przeprowadzony został dokładny opis wykresów wygenerowanych na podstawie danych uzyskanych z symulacji. Przedstawione zostały różnice pomiędzy poszczególnymi protokołami, które działały w określonych warunkach symulacyjnych. Ostatni rozdział zawiera wnioski, które wypływają z analizy wyników symulacji oraz podsumowuje całokształt pracy.

\vskip 2cm


\begin{center}
\large \bf
THESIS TITLE
\end{center}

\section*{Abstract}
This thesis presents a novel way of using a novel algorithm to solve complex
problems of filter design. In the first chapter the fundamentals of filter design
are presented. The second chapter describes an original algorithm invented by the
authors. Is is based on evolution strategy, but uses an original method of filter
description similar to artificial neural network. In the third chapter the implementation
of the algorithm in C programming language is presented. The fifth chapter contains results
of tests which prove high efficiency and enormous accuracy of the program. Finally some
posibilities of further development of the invented algoriths are proposed.

\bigskip
{\noindent\bf Keywords:} thesis, LaTeX, quality

\vfill
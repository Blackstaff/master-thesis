\newpage
\begin{center}
\large \bf
Symulacje wybranych algorytmów trasowania dla bezprzewodowych sieci czujnikowych
\end{center}

\section*{Streszczenie}
Niniejsza praca poświęcona jest symulacjom wybranych algorytmów trasowania w~bezprzewodowych sieciach czujnikowych. Składa się ona z~sześciu rozdziałów. Pierwszy z~nich stanowi wstęp. Zawarte zostały w~nim ogólne informacje dotyczące bezprzewodowych sieci czujnikowych oraz cel pracy. Kolejny z~nich podejmuje tematykę bezprzewodowych sieci czujnikowych w~sposób bardziej szczegółowy. Przedstawiona została definicja sieci czujnikowej, jej budowa oraz zastosowania. Znalazły się tu również opisy wybranych protokołów trasowania, które zostały zaimplementowane oraz zbadane w~ramach pracy. Są to: Flood, SPIN, LEACH, ALEACH i~LEACH-DCHS. Trzeci rozdział przedstawia motywację stojącą za wykorzystaniem oprogramowania do symulacji sieci oraz przedstawia przegląd komercyjnych oraz otwartych rozwiązań w~tej dziedzinie. Szczególna uwaga poświęcona została oprogramowaniu \omnetpp, które zostało wykorzystane podczas tworzenia niniejszej pracy. W~rozdziale pod tytułem ,,Implementacja'' opisane zostały szczegóły implementacyjne poszczególnych protokołów opisanych w~rozdziale drugim. W~ramach pracy powstała również biblioteka modułów \omenetpp, ułatwiająca budowanie symulacji bezprzewodowych sieci czujnikowych w~tymże środowisku. Dodatkowo opisane zostały poprawki błędów w~bibliotece INET bez których nie byłoby możliwe przeprowadzenie symulacji. Rozdział piąty opisuje przeprowadzone symulacje. Przedstawione zostały parametry wejściowe oraz sposób definicji scenariuszy symulacyjnych. Symulacje przeprowadzone zostały dla różnych zmiennych, którymi były: liczba węzłów sieci (czujników), rozmiar danych przesyłanych w~pakiecie z~danymi, czas pomiędzy kolejnymi pakietami z~danymi. Następnie opisany został proces analizy oraz wizualizacji danych. W~celu przetworzenia oraz poprawnego sformatowania danych uzyskanych po przeprowadzeniu symulacji wykorzystany został dedykowany, autorski skrypt napisany w~języku Ruby. Wizualizacja danych została przeprowadzona z~wykorzystanie języka R oraz pakietu ggplot2. W~kolejnych akapitach przeprowadzony został dokładny opis wykresów wygenerowanych na podstawie danych uzyskanych z~symulacji. Przedstawione zostały różnice pomiędzy poszczególnymi protokołami, które działały w~określonych warunkach symulacyjnych. Ostatni rozdział zawiera wnioski, które wypływają z~analizy wyników symulacji oraz podsumowuje całokształt pracy.

\vskip 2cm


\begin{center}
\large \bf
Simulations of Selected Routing Algorithms for Wireless Sensor Networks 
\end{center}

\section*{Abstract}
This thesis presents simulations of selected routing algorithms in wireless sensor networks (WSNs). It consists of six chapters. The first chapter introduces the reader to wireless sensor networks and the aim of the thesis. The next one delves more deeply into the topic of wireless sensor networks. It presents the architecture, design concepts and applications of WSNs. In particular it describes the selected routing protocols which have been implemented and evaluated during the course of this thesis: Flood, SPIN, LEACH, ALEACH and LEACH-DCHS. The third chapter explains the motivation behind the usage of simulation software for wireless sensor network protocol evaluation. Both the commercial and non-commercial are presented. Special attention is given to \omnetpp simulation software, which was used during the research. The following chapter shows the implementation details of the routing algorithms defined in the second chapter. During the course of work on this paper, in order to ease the implementation of the routing algorithms as well as the simulation scenarios, a~library of \omnetpp modules was created. Chapter number five describes the execution of the simulation scenarios. The simulations were carried out with various independent variables: the number of sensor nodes, size of the sensed data, time between consecutive data packets. Then the chapter details the process of analyzing and visualizing the simulations' output data. In order to process and format the data properly, a~dedicated Ruby script was written. The visualization of the data has been created with the help of R programming language and ggplot2 library. Section \ref{sec:analysis} describes the charts which were generated from the resulting output data. They present the performance differences between the routing algorithms. The final chapter closes the thesis by outlining the conclusions based upon the analysis of the simulations' results.    

\vfill
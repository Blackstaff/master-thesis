\documentclass[a4paper,12pt,twoside,openany]{report}
%
% Wzorzec pracy dyplomowej
% J. Starzynski (jstar@iem.pw.edu.pl) na podstawie pracy dyplomowej
% mgr. inż. Błażeja Wincenciaka
% Wersja 0.1 - 8 października 2016
%
\usepackage{polski}
\usepackage{helvet}
\usepackage{xunicode}
\usepackage{fontspec}
\defaultfontfeatures{Mapping=tex-text}
\usepackage{anyfontsize}
\usepackage{xltxtra}
\usepackage{graphicx}
\usepackage{tabularx}
\usepackage{array}
\usepackage[polish]{babel}
\usepackage{subfigure}
\usepackage{amsfonts}
\usepackage{verbatim}
\usepackage{indentfirst}
\usepackage[xetex]{hyperref}
\usepackage[style=numeric]{biblatex}
\usepackage{placeins}
\usepackage{minted}
\usepackage{xcolor}
\usepackage{mdframed}
\usepackage{float}

\maxdeadcycles=900000000

\bibliography{library}

\setminted{autogobble=true}

% Obramowanie dla środowiska "minted"
\surroundwithmdframed[linewidth=1pt, backgroundcolor=white!5]{minted}

% rozmaite polecenia pomocnicze
% gdzie rysunki?
\newcommand{\ImgPath}{./img}

% oznaczenie rzeczy do zrobienia/poprawienia
\newcommand{\TODO}{\textbf{TODO}}


% wyroznienie slow kluczowych
\newcommand{\tech}{\texttt}

\newcommand{\omnetpp}{OMNeT++}

% na oprawe (1.0cm - 0.7cm)*2 = 0.6cm
% na oprawe (1.1cm - 0.7cm)*2 = 0.8cm
%  oddsidemargin lewy margines na nieparzystych stronach
% evensidemargin lewy margines na parzystych stronach
\def\oprawa{1.05cm}
\addtolength{\oddsidemargin}{\oprawa}
\addtolength{\evensidemargin}{-\oprawa}

% table span multirows
\usepackage{multirow}
\usepackage{enumitem}	% enumitem.pdf
\setlist{listparindent=\parindent, parsep=\parskip} % potrzebuje enumitem

%%%%%%%%%%%%%%% Dodatkowe Pakiety %%%%%%%%%%%%%%%%%
\usepackage{prmag2017}   % definiuje komendy opieku,nrindeksu, rodzaj pracy, ...

%%%%%%%%%%%%%%% Strona Tytułowa %%%%%%%%%%%%%%%%%
% To trzeba wypelnic swoimi danymi
\title{
Symulacje wybranych algorytmów trasowania dla bezprzewodowych sieci czujnikowych}

% autor
\author{Mateusz Czarnecki}
\nrindeksu{244593}

\opiekun{dr inż. Łukasz Makowski}
\terminwykonania{1 lutego 2017} % data na oświadczeniu o samodzielności
\rok{2017}


% Podziekowanie - opcjonalne
%\podziekowania{\input{podziekowania.tex}}

% To sa domyslne wartosci
% - mozna je zmienic, jesli praca jest pisana gdzie indziej niz w ZETiIS
% - mozna je wyrzucic jesli praca jest pisana w ZETiIS
%\miasto{Warszawa}
%\uczelnia{POLITECHNIKA WARSZAWSKA}
%\wydzial{WYDZIAŁ ELEKTRYCZNY}
%\instytut{INSTYTUT ELEKTROTECHNIKI TEORETYCZNEJ\linebreak[1] I~SYSTEMÓW INFORMACYJNO-POMIAROWYCH}
 \zaklad{ZAKŁAD SYSTEMÓW INFORMACYJNO-POMIAROWYCH}
%\kierunekstudiow{INFORMATYKA}

% domyslnie praca jest inzynierska, ale po odkomentowaniu ponizszej linii zrobi sie magisterska
\pracamagisterska
%%% koniec od P.W

%\opinie{%
%  \input{opiniaopiekuna.tex}
%  \newpage
%  \input{recenzja.tex}
%}

\streszczenia{
  \newpage
\begin{center}
\large \bf
Symulacje wybranych algorytmów trasowania dla bezprzewodowych sieci czujnikowych
\end{center}

\section*{Streszczenie}
Niniejsza praca poświęcona jest symulacjom wybranych algorytmów trasowania w~bezprzewodowych sieciach czujnikowych. Składa się ona z~sześciu rozdziałów. Pierwszy z~nich stanowi wstęp. Zawarte zostały w~nim ogólne informacje dotyczące bezprzewodowych sieci czujnikowych oraz cel pracy. Kolejny z~nich podejmuje tematykę bezprzewodowych sieci czujnikowych w~sposób bardziej szczegółowy. Przedstawiona została definicja sieci czujnikowej, jej budowa oraz zastosowania. Znalazły się tu również opisy wybranych protokołów trasowania, które zostały zaimplementowane oraz zbadane w~ramach pracy. Są to: Flood, SPIN, LEACH, ALEACH i~LEACH-DCHS. Trzeci rozdział przedstawia motywację stojącą za wykorzystaniem oprogramowania do symulacji sieci oraz przedstawia przegląd komercyjnych oraz otwartych rozwiązań w~tej dziedzinie. Szczególna uwaga poświęcona została oprogramowaniu \omnetpp, które zostało wykorzystane podczas tworzenia niniejszej pracy. W~rozdziale pod tytułem ,,Implementacja'' opisane zostały szczegóły implementacyjne poszczególnych protokołów opisanych w~rozdziale drugim. W~ramach pracy powstała również biblioteka modułów \omenetpp, ułatwiająca budowanie symulacji bezprzewodowych sieci czujnikowych w~tymże środowisku. Dodatkowo opisane zostały poprawki błędów w~bibliotece INET bez których nie byłoby możliwe przeprowadzenie symulacji. Rozdział piąty opisuje przeprowadzone symulacje. Przedstawione zostały parametry wejściowe oraz sposób definicji scenariuszy symulacyjnych. Symulacje przeprowadzone zostały dla różnych zmiennych, którymi były: liczba węzłów sieci (czujników), rozmiar danych przesyłanych w~pakiecie z~danymi, czas pomiędzy kolejnymi pakietami z~danymi. Następnie opisany został proces analizy oraz wizualizacji danych. W~celu przetworzenia oraz poprawnego sformatowania danych uzyskanych po przeprowadzeniu symulacji wykorzystany został dedykowany, autorski skrypt napisany w~języku Ruby. Wizualizacja danych została przeprowadzona z~wykorzystanie języka R oraz pakietu ggplot2. W~kolejnych akapitach przeprowadzony został dokładny opis wykresów wygenerowanych na podstawie danych uzyskanych z~symulacji. Przedstawione zostały różnice pomiędzy poszczególnymi protokołami, które działały w~określonych warunkach symulacyjnych. Ostatni rozdział zawiera wnioski, które wypływają z~analizy wyników symulacji oraz podsumowuje całokształt pracy.

\vskip 2cm


\begin{center}
\large \bf
Simulations of Selected Routing Algorithms in Wireless Sensor Networks
\end{center}

\section*{Abstract}
This thesis presents simulations of selected routing algorithms in wireless sensor networks (WSNs). It consists of six chapters. The first chapter introduces the reader to wireless sensor networks and the aim of the thesis. The next one delves more deeply into the topic of wireless sensor networks. It presents the architecture, design concepts and applications of WSNs. In particular it describes the selected routing protocols which have been implemented and evaluated during the course of this thesis: Flood, SPIN, LEACH, ALEACH and LEACH-DCHS. The third chapter explains the motivation behind the usage of simulation software for wireless sensor network protocol evaluation. Both the commercial and non-commercial are presented. Special attention is given to \omnetpp simulation software, which was used during the research. The following chapter shows the implementation details of the routing algorithms defined in the second chapter. During the course of work on this paper, in order to ease the implementation of the routing algorithms as well as the simulation scenarios, a~library of \omnetpp modules was created. Chapter number five describes the execution of the simulation scenarios. The simulations were carried out with various independent variables: the number of sensor nodes, size of the sensed data, time between consecutive data packets. Then the chapter details the process of analyzing and visualizing the simulations' output data. In order to process and format the data properly, a~dedicated Ruby script was written. The visualization of the data has been created with the help of R programming language and ggplot2 library. Section \ref{sec:analysis} describes the charts which were generated from the resulting output data. They present the performance differences between the routing algorithms. The final chapter closes the thesis by outlining the conclusions based upon the analysis of the simulations' results.    

\vfill
}

\begin{document}
\maketitle

\chapter{Wstęp}
Wzrost potencjału współczesnej nauki jest możliwy dzięki cały czas rozwijającemu się zapleczu technicznemu. To właśnie ono sprawia, że człowiek może widzieć więcej, słyszeć to co do tej pory było niesłyszalne, doświadczać tego co było nieodkryte. Niewątpliwie w rozwój nauk przyrodniczych, medycyny, fizyki i wielu innych miały bezprzewodowe sieci czujników. Pozwoliły one na całodobowy pomiar danej, charakterystycznej dla siebie wielkości.

Dzięki postępom techniki w zakresie energooszczędnych procesorów, komunikacji oraz scalonych układów elektronicznych możliwe stało się stworzenie małych, tanich oraz mających niski pobór energii urządzeń zawierających w sobie dowolny zestaw sensorów oraz aktuatorów. Urządzenia te można łączyć w wielowięzłowe sieci, co zapoczątkowało rozwój dziedziny zajmującej się bezprzewodowymi sieciami czujnikowymi. 

Sieć bezprzewodową charakteryzuje brak sterowania centralnego, który umożliwia rozproszenie dużej ilości czujników w danym, interesującym badacza obszarze. Sieci te same w sobie zapewniają element losowości, który jest bardzo często pożądany w badaniach, ponieważ nie jest możliwe projektowanie dokładnego rozmieszczenia danego czujnika. Najistotniejszą funkcją sieci jest pozyskanie informacji z obszaru, w którym znajdują się sensory.

Bezprzewodowe sieci czujników charakteryzuje szeroki wachlarz zastosowań. Można tu wyróżnić wspomniane wcześniej cele naukowe, a w szczególności te dotyczące badania świata przyrody. Na uwagę zasługują również te sieci, które zostały zamontowane w celach ochronnych takie jak miejski monitoring, czy prewencja włamań do domu, a także systemy ostrzegania o pożarze. Systemy te wykorzystywane są również w miejscach pracy tworząc inteligentne przedszkola czy winnice. Można więc śmiało powiedzieć, że
stały się one częścią naszego życia.

Ponadto bezprzewodowe sieci czujnikowe wykorzystywane są z powodzeniem w prężnie rozwijających się technologiach IoT (Internet of Things). 

Celem pracy było przygotowanie zestawu symulacji wybranych protokołów trasowania WSN oraz przetestowanie ich za pomocą scenariuszy testowych oraz zebranie wyników. Otrzymane wyniki zostały przetworzone, zwizualizowane oraz została przeprowadzona ich analiza.

\chapter{Bezprzewodowe sieci czujnikowe}
Bezprzewodowe sieci czujnikowe są systemami składającymi się dużej liczby węzłów, z których każdy wyposażony w moduły umożliwiające gromadzenie, przetwarzanie oraz przesyłanie informacji \cite{Ilyas2004}, dzięki czemu możliwe staje się obserwowanie oraz reagowanie na zdarzenia występujące w objętym jej zasięgiem środowisku. Środowiskiem tym może mieć charakter zarówno fizyczny, jak też i biologiczny czy stanowić fragment systemu informatycznego. 
\cite{Sohraby2006}
Bezprzewodowa sieć czujnikowa składa się z czterech podstawowych elementów\cite{}:
\begin{enumerate}
	\item Zestawu czujników rozmieszczonych na określonym obszarze
	\item Sieci łączącej te czujniki
	\item Co najmniej jednego węzła gromadzącego dane z pozostałych
	\item Węzłów lub zewnętrznej do sieci jednostki, której zadaniem jest przetworzenie zebranych danych
\end{enumerate}
Wartym odnotowania jest fakt, że przetwarzanie danych może się odbywać w samej sieci czujników, bez konieczności udziału zewnętrznych jednostek.

Sieci czujnikowe są najczęściej same organizują swoją strukturę. Oznacza to, że hierarchia (lub jej brak) sieci oraz trasy pakietów są dynamicznie wypracowywane przez węzły sieci zgodnie z zaimplementowanymi protokołami.
Do innych wyróżniających te sieci charakterystyk należą ograniczenia związane z dostępem do energii elektrycznej, czasem życia baterii, redundancja danych krążących w sieci oraz kolizje pakietów.

Przykładami zastosowań są gromadzenie danych, monitoring, telemetria medyczna \cite{Biradar2009}.
%\section{Czujnik}

Czujnik z technicznego punktu widzenia jest urządzeniem, którego zadaniem jest zbieranie informacji o obiektach i procesach fizycznych wraz ze zmianami ich stanu.\cite{Dargie2010}

W celu stworzenia sieci czujników konieczne jest wcześniejsze zaprojektowanie i stworzenie pojedynczego węzła. Bardzo często narzucone są na nie dodatkowe ograniczenia: wielkość, pobór energii, koszt wytworzenia. Dodatkowo muszą posiadać odpowiednie czujniki, moduły komunikacyjne oraz moc obliczeniową odpowiednią do ich obsłużenia. W książce \cite{Karl2006} architekturę sprzętową pojedynczego węzła rozbito na pięć komponentów:
\begin{enumerate}
	\item Sterownik - komponent odpowiedzialny za przetwarzanie danych oraz wykonywanie instrukcji zawartych w kodzie
	\item Pamięć - komponent przechowujący programy oraz dane
	\item Czujniki i aktuatory - urządzenia zbierające informacje o środowisku zewnętrznym oraz urządzenia wpływające na jego stan. Mogą to być między innymi czujniki wielkości elektrycznych, pola magnetycznego, wykrywacze fal radiowych, czujniki optyczne, elektrooptyczne, podczerwieni, lasery, radary, lokalizacyjne, fal sejsmicznych, badające parametry środowiskowe (pęd wiatru, temperaturę, wilgotność, ciśnienie), biochemiczne \cite{Sohraby2006}.
	\item Komunikacja - komponent odpowiedzialny za wysyłanie oraz odbieranie danych za pośrednictwem bezprzewodowego kanału komunikacyjnego
	\item Zasilanie - najczęściej są to baterie lub ogniwa słoneczne
\end{enumerate}
\section{Zastosowanie}
\section{Trasowanie w WSN}
Bezprzewodowe sieci czujnikowe mają wiele wspólnego z sieciami przewodowymi i ad-hoc, jednakże wykazują również swoje własne, unikatowe cechy oraz związane z nimi wyzwania.
Jedną z takich cech jest różnorodna gęstość sieci oraz liczba węzłów. Sieci czujnikowe mogą składać się z od kilkuset do kilku tysięcy węzłów rozmieszczonych w sposób losowy o różnorodnym zagęszczeniu. Zachowanie tych węzłów jest dynamiczne, jako że do ich zadań oprócz zbierania danych oraz przesyłania pakietów należy również oszczędzanie własnej energii. Dodatkowo duży wpływ na jakość połączenia mają wysokie poziomy szumu oraz interferencja.
Istotnym wyzwaniem związanym z sieciami czujników jest również model danych oraz ich przepływu, który różni się w zależności od konkretnego rozwiązania, w którym sieć jest wykorzystywana. Węzły sieci mogą cyklicznie wysyłać do stacji bazowej próbkę danych. W modelu zdarzeniowym węzeł sieci wysyła pakiet po wystąpieniu określonego zjawiska. Istnieją również implementacje wymagające od węzłów sieci przetwarzania oraz agregacji danych przez węzeł przed ich wysyłką, czy takie, które wymagają komunikacji dwukierunkowej pomiędzy węzłami a stacją bazową.

%Więcej wyzwań z Routing Protocols in Wireless Sensor Networks –
%A Survey

Taka różnorodność modeli przepływu danych oraz innych wyzwań wymagają zastosowania odpowiednio zoptymalizowanych algorytmów trasowania \cite{Abdullah2014, Sohraby2006}.

Z powodu różnych potrzeb odnośnie trasowania pakietów zaproponowany został szereg metryk dotyczących zasobów sieci. Celem protokołów trasowania jest optymalizacja ich wykorzystania \cite{Dargie2010, Biradar2009}.

Minimalizacja ścieżki pakietu do węzła bazowego

Minimalizacja energii zużytej na wysłanie pakietu

Maksymalizacja czasu, po którym następuje podział sieci

Minimalizacja wariancji poziomów energii węzłów


\subsection{Flood}
Węzeł wysyłający pakiet rozgłasza go do najbliższych sąsiadów, którzy z kolei powtarzają ten krok, aż pakiet nie dotrze do wszystkich węzłów sieci lub nie osiągnie maksymalnej liczby skoków.
W przypadku tego algorytmu, jeżeli istnieje droga łącząca źródło pakietu z celem, to cel z pewnością go otrzyma.
Zaletą tego algorytmu jest jego prostota. Do wad natomiast zalicza się duży ruch pakietów w sieci. W celu jego ograniczenia oraz zapewnienia aby pakiet nie był wysyłany w nieskończoność stosowane są dwa mechanizmy \cite{Dargie2010}:
\begin{itemize}
	\item maksymalna liczba przeskoków pakietu
	\item numery sekwencji pakietów - pakiety otrzymują kolejne numery, które wraz z adresem węzła wysyłającego umożliwiają jego identyfikację. Dzięki temu węzły mogą przechowywać historię otrzymanych (oraz rozgłoszonych dalej pakietów) i w momencie w którym taki pakiet ponownie otrzymają - go odrzucić.
\end{itemize}

Mechanizmy te jednakże nie rozwiązują następujących problemów występujących w protokole Flood \cite{Dargie2010}:
\begin{itemize}
	\item Implozja - węzeł, który otrzymał pakiet rozgłasza go do swoich sąsiednich węzłów niezależnie od tego, czy otrzymały one już ten pakiet od innego węzła. Prowadzi to do niepotrzebnego zużycia zasobów. %Dorzucić obrazek
	\item Redundancję geograficzną - pakiety wysyłane przez węzły monitorujące pokrywające się obszary są traktowane jako kompletnie od siebie różne (brak fuzji danych), co prowadzi do marnowania zasobów (ta sama informacji wysyłana jest wielokrotnie). %Dorzucić obrazek
	\item Nieuwzględnianie zasobów węzła - ze względu na swoją prostotę algorytm nie bierze pod uwagę aktualnych zasobów węzła sieci.
\end{itemize}
\subsection{SPIN}
Protokół SPIN (Sensor Protocols for Information via Negotiation) jest rodziną protokołów trasowania typu płaskiego. Do rozsyłania informacji po sieci wykorzystują negocjację. Do ich przeprowadzenia wykorzystują pakiety zawierające metadane opisujące przesyłane wiadomości. Dzięki temu możliwe jest wyeliminowanie redundancji transmisji występujące w protokołach typu Flood \cite{Chaudhary2015}.

Projekt protokołów SPIN wyrósł z protokołów Flood. Twórcy zauważyli trzy podstawowe problemy w tego typu podejściu:
\begin{itemize}
	\item Implozję
	\item Redundancję geograficzną
	\item Nieuwzględnianie zasobów
\end{itemize}

Innowacjami w stosunku do protokołu Flood są negocjacja oraz adaptacja w zależności od zasobów.

\paragraph{Negocjacja} W celu rozwiązania problemów związanych z implozją oraz przenikaniem się monitorowanych obszarów, przed wysłaniem pakietu węzły negocjują między sobą. Do negocjacji wykorzystywane są dodatkowe informacje o pakietach - meta-dane. 
Meta-dane wykorzystywane są do precyzyjnego opisu danych zbieranych przez czujniki. Rozmiar w bajtach meta-danych musi być mniejszy od rozmiaru samego pakietu z danymi, aby rozwiązanie miało sens.
Sam protokół nie narzuca tego co meta-dane powinny zawierać. Jest to zależne od konkretnego rozwiązania oraz implementacji. Może to być np. identyfikator węzła, współrzędne geograficzne, itd. Specyfikacja, przechowywanie oraz przetwarzanie meta-danych wykracza poza algorytm SPIN.
W protokole SPIN węzły komunikują się za pomocą trzech rodzajów pakietów:
\begin{itemize}
	\item ADV - ogłoszenie nowych danych. Jest to pakiet zawierający meta-dane. Ogłasza on węzłom pojawienie się nowych danych
	\item REQ - zgłoszenie zamówienia na dany pakiet z danymi. Węzeł, który chce otrzymać konkretny pakiet wysyła wiadomość z jego meta-danymi (tymi, które otrzymał w ADV)
	\item DATA - pakiet zawierający właściwe dane z czujnika wraz z meta-danymi
\end{itemize}
Jako, że pakiety ADV i REQ zawierają tylko meta-dane, są one tańsze do wysłania niż DATA.

W skład rodziny algorytmów SPIN wchodzą cztery protokoły: SPIN-PP, SPIN-EC, SPIN-BC, SPIN-RL. Jako, że protokoły SPIN-PP oraz SPIN-EC nie przeznaczone są dla sieci point-to-point, opisane zostaną dwa pozostałe protokoły \cite{Dargie2010, Kulik2002}.

Protokół zaczyna się od wysłania pakietu ADV przez węzeł, który posiada nowe dane, które chce rozpropagować w sieci. Węzły sąsiednie, które otrzymały wiadomość ADV, sprawdzają czy otrzymały już takie dane. Jeśli reklamowane dane są dla nich nowe, wysyłają one wiadomość REQ z powrotem do nadawcy. Węzeł inicjujący negocjację po otrzymaniu wiadomości REQ wysyła dane w pakiecie DATA

\subparagraph{SPIN-BC}
W tym wariancie wykorzystany jest fakt, że węzły współdzielą medium komunikacyjne. Komunikacja pomiędzy dwoma węzłami może zostać ''podsłuchana'' przez węzły znajdujące się w zasięgu. Wszystkie pakiety wysyłane są w trybie rozgłoszeniowym, jako że dla sieci bezprzewodowej nie wiąże się to z dodatkowym kosztem.

Wiadomość ADV jest rozgłaszana do wszystkich sąsiednich węzłów. Jednakże przed wysłaniem wiadomości REQ węzły czekają przez losowy okres. Jeżeli węzeł przechwycił wiadmość REQ dotyczącą tych samych danych, których i on potrzebuje, anuluje on wysyłkę swojej wiadomości REQ (nie jest potrzebna, redukuje to koszty). Pozwala to również na uniknięcie kolizji. Po otrzymaniu REQ węzeł wysyła pakiet DATA do kanału rozgłoszeniowego. Pakiet DATA jet wysyłany tylko raz, następujące po nim wiadomości REQ dotyczące tych samych danych są ignorowane.
\subparagraph{SPIN-RL}
SPIN-RL jest przystosowaniem SPIN-BC do warunków komunikacji stratnej. Każdy węzeł przechowuje listę wiadomości ADV i jeśli nie otrzyma on danych w odpowiednim czasie, to ponawia on wysyłkę zapytania o dane (REQ). Odbiorca jest losowany z listy węzłów które wysłały ogłoszenie o tych samych danych.
Po wysyłce pakietu DATA musi upłynąć pewien odstęp czasowy przed ponowną wysyłką.

%Zakończenie
Zgodnie z badaniami opisanymi w artykule \cite{Kulik2002} protokoły SPIN są w stanie dostarczyć 60\% więcej danych w sieciach point-to-point i 80\% więcej danych w sieciach rozgłoszeniowych niż tradycyjne rozwiązania.  
\subsection{LEACH}
LEACH jest protokołem typu hierarchicznego, który wykorzystuje dwa poziomy grupowania węzłów. Pakiety trasowane są od czujników do Cluster Headów i od Cluster Headów do stacji bazowej \cite{Yu2006}.
Węzły same organizują się w klastry, w których jeden węzeł pełni funkcję lokalnego węzła bazowego - nazywany jest on również Cluster Head. W przeciwieństwie do konwencjonalnych algorytmów lokalne węzły bazowe nie są wybierane a priori. Wybierane są rotacyjnie w sposób losowy, tak aby równomiernie rozłożyć zużycie energii oraz wybierać węzły z możliwie jak największą energią. Dodatkowo LEACH wykorzystuje fuzję danych w celu skompresowania ich ilości przesyłanych z klastra do stacji bazowej \cite{Akkaya2005}%\cite{Heinzelman00}.

Działanie algorytmu podzielone jest na rundy. Każda runda składa się z fazy konfiguracyjnej, po której następuje faza stabilnego działania sieci. W fazie początkowej każdy węzeł decyduje o tym czy ma zostać lokalnym węzłem bazowym w aktualnej rundzie. Decyzja ta opiera się na zadanej przez użytkownika liczbie lokalnych węzłów bazowych w sieci (wyrażonej jako procent od liczby wszystkich węzłów sieci) oraz na fakcie uprzedniego bycia lokalnym węzłem bazowym.
Algorytm ten przebiega w sposób następujący:
\begin{enumerate}
	\item Węzeł n losuje liczbę z zakresu od 0 do 1
	\item Obliczany jest próg wyrażony poniższym wzorem
	\[T(n) = \begin{dcases} 
      \frac{P}{1-P*(rmod\frac{1}{P})} & n \in G \\
      0 & n \not\in G \\
   \end{dcases}
	\]
	, gdzie P - procent sieci, którą powinny stanowić lokalne węzły bazowe,
	r - numer aktualnej rundy
	G - zbiór węzłów, które nie były lokalnymi węzłami bazowymi przez ostatnie $\frac{1}{P}$ rund.
	\item Jeżeli wylosowana liczba jest mniejsza od progu $T(n)$, to węzeł zostaje lokalnym węzłem bazowym.
\end{enumerate}
Taki wybór lokalnych węzłów bazowych gwarantuje, że każdy węzeł sieci zostanie nim w ciągu $\frac{1}{P}$ rund. Wszystkie węzły, które zostały wybrane jako lokalne węzły bazowe w pierwszej rundzie (o numerze 0) nie zostaną nimi przez kolejnych $\frac{1}{P}$ rund. Przez kolejne rundy wartość progu $T(n)$ rośnie. Funkcja została zaprojektowana w ten sposób, aby wraz ze zmniejszaniem się liczby węzłów, które mogą zostać potencjalnie wybrane na lokalne węzły bazowe prawdopodobieństwo ich wyboru rosło. Po $\frac{1}{P}$ rundach próg $T(n)$ wraca do stanu początkowego, tym samym rozpoczynając ponownie opisany cykl.
\paragraph{ALEACH}
\paragraph{LEACH DCHS}

\chapter{Narzędzia do symulacji sieci}
% Wykorzystany symulator zdarzeń dyskretnych, opisać po krótce inne
% 1-2 strony
Rozdział ten opisuje narzędzia oraz technologie, które zostały użyte do symulacji działania sieci czujników, implementacji protokołów trasowania oraz przeprowadzenia testów.
\section{Przegląd narzędzi}
\cite{Xian2008}
\cite{Nayyar2015}
\subsection{NS-2}
\subsection{NS-3}
\subsection{IKR}
\subsection{openWNS}
\subsection{Matlab}
\subsection{OPNET}
\subsection{J-Sim}
\subsection{SensorSim}
\subsection{NCTUns}
\subsection{SSFNet}
\subsection{QualNet}
\subsection{SENSE}
\section{\omnetpp}
\omnetpp jest zbudowanym w sposób modularny symulatorem zdarzeń dyskretnych. Stanowi on ogólne narzędzie umożliwiające przeprowadzanie symulacji między innymi: przewodowych oraz bezprzewodowych sieci komputerowych, systemów wieloprocesorowych, chmur obliczeniowych czy też ruchu miejskiego. Dla każdej bardziej wyspecjalizowanej dziedziny konieczne jest stworzenie nowego modelu (zbioru modułów) lub skorzystanie z jednego z już istniejących rozwiązań (np. INET, VEINS).\cite{Varga2017}
\subsection{Moduły}
Podstawowym budulcem symulacji w \omnetpp są moduły. Dzielą się one na proste oraz złożone.
% Opis co to moduł oraz że są proste i złożone, możne je zagneżdżać oraz wiązać za pomocą bram (ang. gate). A komunikują się za pomocą wiadomości i sygnałów

Do implementacji modułów wykorzystuje się dwa języki
\begin{enumerate}
	\item NED - język domenowy \omnetpp, za pomocą którego definiowane są moduły. Zawierają opis parametrów oraz bram modułu. Dodatkowo w przypadku modułów złożonych definiowane są zagnieżdżone moduły wraz z ich połączeniami.
	%Wstawić przykład
	\item C++ - wykorzystywany do implementacji modułów prostych (moduły złożone definiowane są tylko za pomocą plików NED).
	%Wstawić przykład
\end{enumerate}
Dodatkowo, w celu przyspieszenia oraz ułatwienia implementacji wiadomości, które moduły wykorzystują w komunikacji stworzono odpowiedni język dziedzinowy, który tłumaczony jest przez kompilator do kodu C++.
%Wstawić przykład

Moduły umieszczane są w Sieciach (Network), które również zdefiniowane są w języku NED.

Do komunikacji moduły wykorzystują zdefiniowane (w pliku .ned) przez programistę łącza (ang. links ) oraz bramy (ang. gates).
\subsection{Przeprowadzenie symulacji}
\omnetpp umożliwia skorzystanie z różnych interfejsów użytkownika: QT, TKenv oraz wiersza poleceń. W fazie testowania poprawności przygotowanej symulacji skorzystać można z jednego z interfejsów graficznych. Po weryfikacji zalecana jest rezygnacja z interfejsu graficznego w celu przyspieszenia działania symulacji. Niebywałym ułatwieniem w przygotowaniu oraz uruchomieniu zestawu symulacji w celu zebrania danych do analizy są pliki konfiguracyjne. Umożliwiają one określenie parametrów modułów wchodzących w skład sieci oraz zadeklarowanie parametrów wchodzących w skład zmiennych badanych w ramach symulacji (parameter studies).
%Wstawić przykład
\section{INET}
INET jest biblioteką/modelem zawierającym implementacji wielu protokołów sieciowych (m.in. IPv4, IPv6, TCP, SCTP, UDP), jak również standardów komunikacji wykorzystywanych m.in. w sieciach czujników (IEEE 802.15.4).\cite{inet}

%Opisać, że biblioteka zawierała błędy, opisać jakie błędy i jak zostały naprawione

\chapter{Implementacja}
\section{Framework}
W ramach pracy powstała minibiblioteka zawierająca moduły ułatwiające implementację protokołów działających w ramach WSN. Stanowi ona rozszerzenie biblioteki INET, zawierającej moduły dla standardu IEEE 802.15.4.

Biblioteka składa się z następujących modułów:
\begin{itemize}
	\item WSNNode --- złożony moduł ,,abstrakcyjny'' stanowiący bazę dla węzłów sieci. W jego skład wchodzą:
\paragraph{Moduł mobilności} Jest to moduł zarządzający położeniem oraz mobilnością węzła. Jako, że zakres pracy obejmuje węzły stacjonarne, jako wartość domyślna został przyjęty moduł StationaryMobility, który jedynie śledzi położenie węzła.
\paragraph{Źródło energii} Jako źródło energii wykorzystany został udostępniony przez Inet moduł prostego zasobnika energii (SimpleEnergyStorage).
\paragraph{Tablica interfejsów} Przechowuje informacje o interfejsach danego węzła. W przypadku objątych tą pracą symulacji jest to jeden interfejs radiowy.
\paragraph{Moduł stanu węzła} Jest to moduł informujący o aktualnym stanie węzła.
\paragraph{Moduł monitora} Jest to autorski moduł monitorujący poziom energii w węźle do celów statystycznych. Moduł w regularnych odstępach czasu emituje sygnał zawierający aktualną ilość energii w węźle.
\paragraph{Moduł warstwy sieciowej} Odpowiada za warstwę sieciową węzła. Zawierają się w nim moduły odpowiedzialne za trasowanie pakietów.
\paragraph{Moduł karty sieciowej} Jest to moduł symulujący kartę sieciową. Zawiera się w nim moduł odpowiadający za warstwę fizyczną oraz moduł łącza danych.
\begin{figure}[!htbp]
	\begin{center}
		\centering
		\includegraphics[scale=1]{\ImgPath/framework/node.png} 
	\end{center}
	\caption{Węzeł sieci}
	\label{abstractNode}
\end{figure}
\FloatBarrier
	\item WSNSensorNode - dziedziczy po WSNNode oraz zawiera dodatkowo moduł generujący pakiety
	\begin{figure}[!htbp]
	\begin{center}
		\centering
		\includegraphics[scale=1]{\ImgPath/framework/sensor.png} 
	\end{center}
	\caption{Czujnik}
	\label{openlayers}
\end{figure}
\FloatBarrier
	\item WSNSinkNode - dziedziczy po WSNNode oraz zawiera dodatkowo moduł akcetujący pakiety od czujników
	\begin{figure}[!htbp]
	\begin{center}
		\centering
		\includegraphics[scale=1]{\ImgPath/framework/sink.png} 
	\end{center}
	\caption{Stacja bazowa}
	\label{abstractNode}
\end{figure}
\FloatBarrier
	\item NodeCounter - moduł liczący działające węzły w sieci
	\item SimpleEnergyConsumer - moduł pobierający energię na żądanie
	\item VolatileStateBasedEnergyConsumer - moduł pobierający energię w zależności od ustawionego stanu z możliwością dynamicznej zmiany poboru mocy
\end{itemize}
\section{Protokoły}
Każda implementacja wybranego algorytmu trasowania wymaga stworzenia dwóch modułów. Jeden z nich dziedziczy po SimpleNetworkLayer, a drugi po NetworkProtocolBase.

Głównymi funkcjami, które jednocześnie wywoływane są przez silnik symulacji są handleSelfMessage, handleUpperPacket oraz handleLowerPacket.

\begin{minted}{cpp}
    virtual void handleSelfMessage(cMessage *msg) override;
    /** @brief Handle messages from upper layer */
    virtual void handleUpperPacket(cPacket *) override;

    /** @brief Handle messages from lower layer */
    virtual void handleLowerPacket(cPacket *) override;
\end{minted}
% flood jako akapit
\subsection{Flood}
W implementacji protokołu Flood wykorzystany został istniejący już moduł z biblioteki INET. Należało jedynie utworzyć  moduł będący warstwą sieciową, a następnie skonfigurować go, aby wykorzystywał moduł Flood z biblioteki INET.
\subsection{SPIN}
Protokół SPIN został zaimplementowany zgodnie z zaproponowanym w artykule \cite{Kulik2002} wariancie SPIN-RL.

Kod odpowiadający za logikę protokołu znajduje się w klasie SPIN, która dziedziczy po klasie NetworkProtocolBase oraz implementuje interfejs INetworkProtocol z biblioteki INET.

\begin{minted}{cpp}
class SPIN : public NetworkProtocolBase, public INetworkProtocol
\end{minted}

Moduł po otrzymaniu pakietu z wyższej warstwy dokonuje decyzji, czy węzeł ma dostatecznie dużo energii, aby przeprowadzić negocjację. W przypadku pozytywnej decyzji rozpoczynany jest proces negocjacji. W przeciwnym wypadku dane rozgłaszane są bezpośrednio, z pominięciem negocjacji.

\begin{minted}{cpp}
void SPIN::handleUpperPacket(cPacket *m)
{
    SPINDatagram *msg = encapsMsg(m, DATA);
    msg->setSeqNum(seqNum);
    seqNum++;

    if (isNegotiationViable()) {
        advertiseData(msg);
    } else {
        simpleSend(msg);
    }
    nbDataPacketsSent++;
}
\end{minted}

Algorytm decyzyjny przedstawiony jest na poniższym listingu.

\begin{minted}{cpp}
bool SPIN::isNegotiationViable()
{
    double randomNumber = uniform(0, 1);
    double currentEnergy = energyStorage->getResidualCapacity().get();
    double k = -1.2;
    double currentEnergyFrac = currentEnergy / maxEnergy;

    return randomNumber < (k*currentEnergyFrac / (k - currentEnergyFrac + 1));
}
\end{minted}

\begin{minted}{cpp}
std::map<MsgMetadata, SPINDatagram*, MsgMetadataCompare> 
    queuedMessages;
std::map<MsgMetadata, SPINDatagram*, MsgMetadataCompare> 
    queuedRequests;
std::set<MsgMetadata, MsgMetadataCompare> knownMessages;
std::set<MsgMetadata, MsgMetadataCompare> requestedMessages;
\end{minted}

\begin{minted}{cpp}
    void advertiseData(SPINDatagram *msg);
    void scheduleReq(MsgMetadata metadata, L3Address advertiser);
\end{minted}
\subsection{LEACH}
\begin{minted}{cpp}
void LEACH::handleUpperPacket(cPacket *m)
{
    if (!isSink) {
        LEACHPacket *netPacket = encapsMsg(m, LEACH_DATA_PACKET);
        if (!isCH && endFormClus) {
            CHInfo info = *CHcandidates.begin();
            netPacket->setDestAddr(info.src);
            bufferPacket(netPacket);
        } else if (!isCH && !endFormClus) {
            tempTXBuffer.push(netPacket);
        } else if (isCH) {
            bufferAggregate.push_back(netPacket);
        }
    }
}
\end{minted}

\begin{minted}{cpp}
case LEACH_DATA_PACKET:
            dest = msg->getDestAddr();
            if (isCH && interfaceTable->isLocalAddress(dest)) {
                bufferAggregate.push_back(msg);
            } else if (interfaceTable->isLocalAddress(dest) && isSink) {
                unpackAndSendUp(msg);
            } else {
                delete msg;
            }
break;
\end{minted}

\begin{minted}{cpp}
case LEACH_ADV_PACKET:
            if (!isCH && !isSink) {
                CHInfo rec;
                rec.src = msg->getSrcAddr();
                rec.rssi = rssi;
                CHcandidates.push_front(rec);
            }
            delete msg;
break;
\end{minted}

\begin{minted}{cpp}
case LEACH_JOIN_PACKET:
            dest = msg->getDestAddr();
            if (isCH && interfaceTable->isLocalAddress(dest)) {
                clusterMembers.push_back(msg->getSrcAddr());
            }
            delete msg;
break;
\end{minted}

\begin{minted}{cpp}
case LEACH_TDMA_PACKET:
            if (!isCH && !isSink) {
                clusterLength = msg->getScheduleArraySize();
                slotLength = roundLength*0.9 / clusterLength;
                for (int i = 0; i < clusterLength; i++) {
                    if (msg->getSchedule(i) == myNetwAddr) {
                        setStateSleep();
                        setTimer(START_SLOT, i * slotLength);
                        break;
                    }
                }
            }
            delete msg;
break;
\end{minted}
\subsection{ALEACH}
\begin{minted}{cpp}
void ALEACH::selectCH()
{
    if (roundNumber >= (numSensors / expectedCHNum)) {
        roundNumber = 0;
        isCt = false;
        isCH = false;
    }

    double randomNumber = uniform(0, 1);
    if (isCH) {
        isCH = false;
        isCt = true;
    }
    double generalProb = (double)expectedCHNum / (double)(numSensors - expectedCHNum * (roundNumber % (numSensors / expectedCHNum) ));
    double currentEnergy = energyStorage->getResidualCapacity().get();
    double currentStateProb = (currentEnergy / maxEnergy) * ((double) expectedCHNum / numSensors);
    if (isCt) {
        probability = 0;
    } else {
        probability = generalProb + currentStateProb;
    }
    if (randomNumber < probability) {
        isCH = true;
    }
}
\end{minted}
\subsection{LEACH DCHS}
\begin{minted}{cpp}
void LEACH_DCHS::selectCH()
{
    if (roundNumber >= (1 / percentage)) {
        roundNumber = 0;
        isCt = false;
        isCH = false;
    }

    double randomNumber = uniform(0, 1);
    if (isCH) {
        isCH = false;
        isCt = true;
    }
    double currentEnergy = energyStorage->getResidualCapacity().get();
    if (isCt) {
        probability = 0;
    } else {
        probability = percentage / (1 - percentage * (roundNumber % (int)(1/percentage)))
                * (currentEnergy / maxEnergy + (notCHRounds / (1/percentage)) * (1 - (currentEnergy / maxEnergy)));
    }
    if (randomNumber < probability) {
        isCH = true;
        notCHRounds = 0;
    } else if (!isCt) {
        ++notCHRounds;
    }
}
\end{minted}
\section{Poprawa biblioteki INET}
Umożliwienie warstwie sieciowej dostępu do rssi. Problem został już wcześniej zasygnalizowany przez jednego z użytkowników, jednakże nie został on rozwiązany. Znajomość rssi jest niezbędna do prawidłowego działania algorytmów opartych na LEACH.

Dodanie możliwości zmiany modułu MAC w Ieee802154NarrowbandNic.

Poprawa modułu IPvXTrafGen, tak aby prawidłowo zachowywał się podczas deaktywacji węzła.

Uzupełnienie modułu CSMA o funkcje obsługujące jego wyłączenie oraz ponowne włączenie. Przed poprawką, wyłącznie się symulowanego węzła (w wyniku zużycia energii) powodowało wystąpienie wyjątku i awaryjne zakończenie symulacji.

Obsługę przypadku, gdy pakiet dociera do wyłączonego już węzła.

Przeniesienie trzech poprawek autorstwa Floriana Kauera dotyczących szczegółów związanych z danymi zaczerpniętymi ze specyfikacji mikrokontrolera na podstawie której zdefiniowano domyślne parametry w Inecie.

\chapter{Symulacje}
W ramach pracy został przeprowadzony zestaw scenariuszy symulacyjnych. Każdy z zaimplementowanych protokołów został przetestowany z przygotowanymi zestawami parametrów wejściowych.
Do tych parametrów należą:
\paragraph{Dystrybucja węzłów sieci}
Symulacje zostały przeprowadzone dla dwóch sposobów dystrybucji węzłów: zgodnej z rozkładem normalnym oraz zgodnej z rozkładem jednorodnym.

Dla rozkładu normalnego średnia współrzędnej wynosiła 400m, a jej odchylenie standardowe 100m.

\begin{figure}[!htbp]
	\begin{center}
		\includegraphics[scale=0.35]{\ImgPath/charts/normal_distribution.png}
	\end{center}
	\caption{Przykładowa dystrybucja węzłów sieci zgodna z rozkładem normalnym}
\end{figure}

Z rozkładem jednorodnym węzły zostały rozłożone na obszarze o kształcie prostokąta i wymiarach 450m na 550m.

\begin{figure}[!htbp]
	\begin{center}
		\includegraphics[scale=0.35]{\ImgPath/charts/uniform_distribution.png}
	\end{center}
	\caption{Przykładowa dystrybucja węzłów sieci zgodna z rozkładem jednorodnym}
\end{figure}

\paragraph{Liczba węzłów sieci}
Symulacje przeprowadzono dla sieci składającej się z dwudziestu oraz dwustu czujników.
\paragraph{Rozmiar pakietu z danymi}
Symulacje przeprowadzono dla rozmiarów pakietów 5B, 50B, 500B, 5000B.
\paragraph{Częstotliwość generowania nowych danych}
Symulacje przeprowadzono dla okresów: 5s, 10s, 15s, 20s.
\paragraph{Początkowa wartość energii elektrycznej w węźle}
Jako początkową wartość dla czujnika przyjęto 5J.
%uzasadnić na krzywej zużycia baterii

Dodatkowo dla protokołów z rodziny LEACH zdefiniowane zostały parametry:
\paragraph{Procent lokalnych węzłów bazowych w sieci}
\paragraph{Długość rundy}
%\paragraph{Współczynnik kompresji danych przez cluster heady}

W celu zniwelowania wpływu losowego rozmieszczenia węzłów na długość działania sieci, dla każdego pojedynczego zestawu parametrów wykonano 5 przebiegów symulacji z różnym ziarnem. Wyniki tych przebiegów zostały uśrednione.

Na poniższym listingu znajduje się fragment pliku konfiguracyjnego symulacji.
\begin{verbatim}
**.numSensors = ${numSensors = 20, 200, 2000}
**.app.protocol = 59
**.app.packetLength = ${packetLength = 5, 50, 500, 5000}B
**.app.startTime = 1s + exponential(10ms)
**.networkLayerType = "FloodNetworkLayer"
**.sink.protocol = 59
**.broadcastSinkAddress = false
**.sink.nic.mac.address = "FF-FF-FF-FF-FF-FE"
**.app.destAddresses = "FF-FF-FF-FF-FF-FE"
**.np.sinkAddr = "FF-FF-FF-FF-FF-FE"
**.radio.displayCommunicationRange = true
**.app.sendInterval = ${sendInterval = 5, 10, 15, 20}s + exponential(10ms)

**.sensor[*].energyStorage.nominalCapacity = ${nodeEnergy = 5}J
**.sink.energyStorage.nominalCapacity = 4J * ${nodeEnergy}
**.energyStorage.nodeShutdownCapacity = ${nodeEnergy} * 0.1J
**.energyStorage.nodeStartCapacity = ${nodeEnergy} * 0.2J
\end{verbatim}

Dla każdej symulacji zostały również zdefiniowane stałe warunki środowiskowe:
\begin{itemize}
	\item Przestrzeń - idealnie płaski teren bez żadnych przeszkód.
	\item Prędkość propagacji fali - założono model propagacji fali ze stałą prędkością. Oznacza to, że czas propagacji jest proporcjonalny do przebytej odległości zdeterminowanej przez stały parametr prędkości.
	\item Tłumienie trasy radiowej - założono model Breakpoint path loss.
	\item Szum tła - założono szum izotropiczny o mocy -96.616dBm.
	\item Do reprezentacji sygnału analogowego wykorzystany został model skalarny - opisuje on sygnał za pomocą jego mocy skalarnej, szerokości pasma oraz częstotliwości fali nośnej.
\end{itemize}

\section{Analiza i wizualizacja danych}
Uruchomienie wszystkich scenariuszy symulacji spowodowało wygenerowanie dużej ilości danych ($\sim$300GB) rozłożonej na tysiące plików. W celu analizy takiej ilości danych napisany został skrypt w języku Ruby \cite{ruby}, którego zadaniem była decymacja oraz fuzja wszystkich plików.

Do analizy oraz wizualizacji danych wykorzystany został język R wraz z pakietem ggplot2. Dla ułatwienia porównań wykresy zostały zebrane w większe macierze i przedstawione w załączniku. Dla każdej topologii sieci powstały dwie takie macierze: jedna zawierająca wykresy liczby aktywnych czujników, druga natomiast zawiera wykresy przedstawiające sumę energii w całej sieci. Na wykresach energii sieci za pomocą przerywanych pionowych linii oznaczony został czas, po którym nastąpiło wyłączenie pierwszego węzła w sieci z powodu zbyt niskiego jej poziomu (poniżej 10\%). Od tego momentu uważa się, że sieć działa w trybie niestabilnym.
Każdy protokół został przetestowany z następującymi parametrami wejściowymi:
\begin{itemize}
	\item Rozmiar pakietu danych: 5B, 50B, 500B, 5000B
	\item Czas po którym następuje wygenerowanie nowego pakietu danych: 5s, 10s, 15s, 20s
	\item Liczba węzłów sieci: 20, 200
\end{itemize}

Dodatkowo, po przeprowadzeniu wszystkich scenariuszy testowych oraz analizie wyników okazało się, że protokół SPIN oraz Flood uzyskały bardzo zbliżone rezultaty, a co za tym idzie ich wykresy pokrywają się w każdym przypadku. Wszelkiego rodzaju spostrzeżenia i obserwacje (zawarte w opisach wykresów) dotyczące protokołu SPIN będą się odnosić również do protokołu Flood. 
W przypadku tych protokołów dezaktywacja węzłów sieci przebiega gwałtownie oraz lawinowo - większość węzłów sieci zostaje wyłączonych w okolicach jednego punktu w czasie.

\newpage
Na wykresach została przedstawiona zależność pomiędzy liczbą aktywnych węzłów w zależności od odstępów pomiędzy kolejnymi pakietami.Sieć składała się z dwudziestu węzłów, które rozmieszczone zostały zgodnie z rozkładem normalnym.

Dla pakietów o rozmiarze 5B można zauważyć najdłuższy czas działania sieci do zakończenia pracy ostatniego węzła. Dla tego pakietu najdłużej działającymi protokołami były ALEACH i LEACH DCHS. Zależność ta była najbardziej widoczna w przypadku odstępów równych 20s. Na uwagę zasługuje również wzrost czasu działania sieci wraz ze wzrostem przerwy pomiędzy okresami.

\begin{figure}[H]
	\begin{center}
		\includegraphics[scale=0.5]{\ImgPath/charts/alive_nodes_normal_20sensors_row1.png}
	\end{center}
	\caption{Aktywne węzły - liczba czujników: 20, rozmiar pakietu danych: 5B, rozkład normalny}
\end{figure}

Wykresy pakietów 50B i 500B wydają się relatywnie podobne, jednak można zauważyć różnice pomiędzy nimi. Główną różnicą jest dłuższy czas działania sieci w przypadku pakietów 50B. Wykresy te prezentują znaczne podobieństwa w przebiegu protokołów, wyjątkiem jest protokół ALEACH. W wykresach dla pakietów 50B z odstępami równymi 10s i 20s działał on zdecydowanie najdłużej. W przypadku wykresu z odstępami 20s działanie było dłuższe o ponad 50 sekund. Tak dobrych wyników protokół ten nie osiągał już w przypadku pakietu 500B. W tym przypadku widać podobieństwo w długości działania sieci. Rozbieżności są widoczne jedynie na wykresie o odstępach 20s, gdzie najdłużej działającym protokołem był LEACH DCHS. 

\begin{figure}[H]
	\begin{center}
		\includegraphics[scale=0.5]{\ImgPath/charts/alive_nodes_normal_20sensors_row2.png}
	\end{center}
	\caption{Aktywne węzły - liczba czujników: 20, rozmiar pakietu danych: 50B, rozkład normalny}
\end{figure}

\begin{figure}[H]
	\begin{center}
		\includegraphics[scale=0.5]{\ImgPath/charts/alive_nodes_normal_20sensors_row3.png}
	\end{center}
	\caption{Aktywne węzły - liczba czujników: 20, rozmiar pakietu danych: 500B, rozkład normalny}
\end{figure}

Dla pakietów o rozmiarze 5000B można zauważyć najkrótszy czas działania wszystkich pakietów. Dla tych, w których przerwa wynosiła 5s, 10s i 15s ostatni węzeł kończył pracę już przed 200 sekundą. Warto nadmienić, że dla pakietów z przerwami 5s i 20s protokołem działającym najdłużej był ALEACH. 

\begin{figure}[H]
	\begin{center}
		\includegraphics[scale=0.5]{\ImgPath/charts/alive_nodes_normal_20sensors_row4.png}
	\end{center}
	\caption{Aktywne węzły - liczba czujników: 20, rozmiar pakietu danych: 5000B, rozkład normalny}
\end{figure}
 

Na wykresach przedstawiono zależność pomiędzy sumą energii sieci w zależności od czasu, rozmiaru  i rodzaju pakietu. Sieć składała się z 20 węzłów początkowych rozmieszczonych zgodnie z rozkładem normalnym.
  
Ogólną tendencją, którą można zauważyć na wykresach najwolniejszy spadek energii zgromadzonej w sieci dla pakietów wysyłanych w odstępach 20s. Sieć najszybciej traci energię dla protokołów SPIN oraz Flood. Protokołem, który najwolniej wytracał energię, jest ALEACH, jednak ta tendencja nie ma zastosowania we wszystkich wykresach.

Dla pakietów równych 5B można zaobserwować najdłuższy wykres, najwolniejszy spadek energii sieci. Wykres ten był przypisany do pakietów danych wysyłanych z odstępami o wartości 20s. W tym przypadku najdłużej działającymi protokołami były ALEACH i LEACH DCHS. Na wykresach dla pakietów danych o rozmiarze 5B można zaobserwować relatywnie dużą różnicę pomiędzy SPIN a pozostałymi protokołami. Ten pierwszy zdecydowanie szybciej wytraca sumę energii.

\begin{figure}[H]
	\begin{center}
		\includegraphics[scale=0.5]{\ImgPath/charts/stored_energy_normal_20sensors_row1.png}
	\end{center}
	\caption{Energia sieci - liczba czujników: 20, rozmiar pakietu danych: 5B, rozkład normalny}
\end{figure}

Wykresy pakietów o rozmiarach 50B i 500B wydają się podobne. Różnicą, którą można zaobserwować jest zmniejszenie się rozbieżności pomiędzy SPIN, a pozostałymi protokołami. Najwyraźniej zmianę tą obserwuje się w pakietach wysyłanych co 5s dla rozmiarów 50B i 500B. Tendencją w wykresach pakietów o rozmiarze 500B jest zwiększanie się różnicy między SPIN, a pozostałymi protokołami wraz ze wzrostem odstępu między wysyłką kolejnych pakietów danych. Na uwagę zasługuje również wykres dla pakietów o rozmiarze 50B wysyłanych w odstępie 20s. Zauważono, że sieć użyjąca protokołu ALEACH wyraźnie dłużej wytraca energię niż w przypadku pozostałych protokołów. 

\begin{figure}[H]
	\begin{center}
		\includegraphics[scale=0.5]{\ImgPath/charts/stored_energy_normal_20sensors_row2.png}
	\end{center}
	\caption{Energia sieci - liczba czujników: 20, rozmiar pakietu danych: 50B, rozkład normalny}
\end{figure}

\begin{figure}[H]
	\begin{center}
		\includegraphics[scale=0.5]{\ImgPath/charts/stored_energy_normal_20sensors_row3.png}
	\end{center}
	\caption{Energia sieci - liczba czujników: 20, rozmiar pakietu danych: 500B, rozkład normalny}
\end{figure}

Wykresami, które różnią się od pozostałych są te o rozmiarze pakietu danych wynoszącym 5000B. Nie obserwuje się tu różnicy pomiędzy SPIN, a pozostałymi protokołami. Wyjątkiem jest wykres dla pakietu z okresami odstępu 20s, gdzie SPIN nieznacznie szybciej zużywa energię sieci. Zauważono, że ogólna suma energii węzłów dla pakietów o rozmiarze 5000B jest szybciej wytracana niż w przypadku pozostałych wykresów.

\begin{figure}[H]
	\begin{center}
		\includegraphics[scale=0.5]{\ImgPath/charts/stored_energy_normal_20sensors_row4.png}
	\end{center}
	\caption{Energia sieci - liczba czujników: 20, rozmiar pakietu danych: 5000B, rozkład normalny}
\end{figure}

Na wykresach przedstawiono liczbę aktywnych czujników w zależności od rozmiaru pakietu, okresami między pakietami i rodzajem protokołu. Sieć składała się z dwustu węzłów, które rozmieszczone zostały zgodnie z rozkładem normalnym.

Ogólnym trendem, który można zauważyć wśród wykresów jest szybsze zmniejszanie się liczby aktywnych czujników wraz ze wzrostem rozmiaru pakietu. Można zauważyć, że często najdłużej czujniki pozostawały aktywne w czasie pracy protokołu LEACH DCHS oraz LEACH. Inną istotną zależnością jest wzrost liczby aktywnych czujników skorelowany ze wzrostem okresu między pakietami.
 
Dla rozmiaru pakietu 5B można zaobserwować najwyższe wartości aktywnych czujników, które dodatkowo rosną wraz ze zwiększającym się okresem między pakietami. Na uwagę zasługuje wykres, w którym pakiety danych wysyłane są co 10s. Tu największa liczba czujników po 300 sekundzie była odnotowana w przypadku zastosowania protokołu LEACH. Na pozostałych wykresach tej grupy najwyższą liczbę czujników odnotowano dla protokołu LEACH DCHS.

\begin{figure}[H]
	\begin{center}
		\includegraphics[scale=0.5]{\ImgPath/charts/alive_nodes_normal_200sensors_row1.png}
	\end{center}
	\caption{Aktywne węzły - liczba czujników: 200, rozmiar pakietu danych: 5B, rozkład normalny}
\end{figure}


Podobne zależności odnotowano w przypadku rozmiaru pakietu 50B. Wyjątkiem jest wykres o okresach między pakietami równymi 5s. Tu czujniki wykorzystujące protokół LEACH DCHS szybciej ulegały dezaktywacji niż w przypadku pozostałych protokołów. 

\begin{figure}[H]
	\begin{center}
		\includegraphics[scale=0.5]{\ImgPath/charts/alive_nodes_normal_200sensors_row2.png}
	\end{center}
	\caption{Aktywne węzły - liczba czujników: 200, rozmiar pakietu danych: 50B, rozkład normalny}
\end{figure}


Można zaobserwować zmianę pewnych tendencji w przypadku pakietów o rozmiarze 500B. Tu szybciej zmniejszała się liczba aktywnych czujników, a sama długość ich działania jest krótsza niż w przypadku pakietów o rozmiarze 5B i 50B. Warto zwrócić uwagę na wykres o okresie między pakietami równym 10s. Tu najdłużej aktywnymi czujnikami były te, w których zastosowano protokół LEACH i ALEACH.

\begin{figure}[H]
	\begin{center}
		\includegraphics[scale=0.5]{\ImgPath/charts/alive_nodes_normal_200sensors_row3.png}
	\end{center}
	\caption{Aktywne węzły - liczba czujników: 200, rozmiar pakietu danych: 500B, rozkład normalny}
\end{figure}

 
Wykresy protokołów dla rozmiaru 5000B są zdecydowanie najkrótsze. Na uwagę zasługuje ten z okresami między pakietami równymi 5s, dla którego czas działania sieci nie przekracza 200 sekund. Najdłuższy wykres z tej grupy, w którym przerwy były równe 20s, opisuje sieć działającą niecałe 400 sekund, gdzie w pozostałych protokołach dla tej długości okresów między pakietami czas ustania aktywności ostatniego czujnika przypadał na około 600 sekund od czasu rozpoczęcia. Istotne jest również, to że w tej grupie najszybciej spadała liczba aktywnych czujników.

\begin{figure}[H]
	\begin{center}
		\includegraphics[scale=0.5]{\ImgPath/charts/alive_nodes_normal_200sensors_row4.png}
	\end{center}
	\caption{Aktywne węzły - liczba czujników: 200, rozmiar pakietu danych: 5000B, rozkład normalny}
\end{figure}
 

Na wykresach przedstawiono sumę energii wyrażoną w dżulach w zależności od czasu, rozmiaru pakietu, długości okresów między pakietami i rodzajem protokołu. Sieć składała się z dwustu węzłów, które rozmieszczone zostały zgodnie z rozkładem normalnym.

Największą różnicę pomiędzy protokołem SPIN, a pozostałymi protokołami można zauważyć w pakietach o rozmiarze 5B. SPIN dużo szybciej zużywa energię sieci, a jego wykres kończy się zazwyczaj w pierwszej połowie długości pozostałych protokołów. Tendencja, która jest zauważalna w tej grupie to wydłużający się czas zużycia energii wraz ze wzrostem okresów między pakietami. Warto nadmienić, że protokoły ALEACH, LEACH i LEACH DCHS miały relatywnie podobny przebieg.

\begin{figure}[H]
	\begin{center}
		\includegraphics[scale=0.5]{\ImgPath/charts/stored_energy_normal_200sensors_row1.png}
	\end{center}
	\caption{Energia sieci - liczba czujników: 200, rozmiar pakietu danych: 5B, rozkład normalny}
\end{figure}

Długości wykresów dla pakietów o rozmiarze 50B są podobne do tych o rozmiarze 5B. Można w nich zauważyć tożsame tendencje ogólne. Warto zwrócić uwagę na zmiany, jakie w poszczególnych wykresach prezentuje protokół LEACH DCHS. Dla okresów między pakietami wynoszącym 5s, w drugiej połowie swojej długości szybciej od innych protokołów zaczął wytracać energię, natomiast dla okresów między pakietami o długości 20s najwolniej wytracał energię.

\begin{figure}[H]
	\begin{center}
		\includegraphics[scale=0.5]{\ImgPath/charts/stored_energy_normal_200sensors_row2.png}
	\end{center}
	\caption{Energia sieci - liczba czujników: 200, rozmiar pakietu danych: 50B, rozkład normalny}
\end{figure}

W opisanych poprzednio wykresach można zaobserwować zmianę w tych, dla których rozmiar pakietów danych wynosił 500B. Różnica między protokołem SPIN, a pozostałymi protokołami jest mniejsza, jednak zwiększa się ona wraz ze wzrostem długości odstępu pomiędzy kolejnymi pakietami z danymi. Ogólną tendencją, która wystąpiła w wykresach dla pakietów danych o rozmiarze 500B jest szybsze wytracania energii. Jest to szczególnie widoczne, gdy przerwa między pakietami jest krótsza.

\begin{figure}[H]
	\begin{center}
		\includegraphics[scale=0.5]{\ImgPath/charts/stored_energy_normal_200sensors_row3.png}
	\end{center}
	\caption{Energia sieci - liczba czujników: 200, rozmiar pakietu danych: 500B, rozkład normalny}
\end{figure}

Najbardziej odbiegające od pierwszych wykresów są te, które przedstawiają pakiety o rozmiarze 5000B. Tu protokoły ALEACH, LEACH i LEACH DCHS bardzo zbliżyły się do sumy energii sieci używającej protokołu SPIN. Wykresy tej grupy są też najkrótsze w całej macierzy.

\begin{figure}[H]
	\begin{center}
		\includegraphics[scale=0.5]{\ImgPath/charts/stored_energy_normal_200sensors_row4.png}
	\end{center}
	\caption{Energia sieci - liczba czujników: 200, rozmiar pakietu danych: 5000B, rozkład normalny}
\end{figure}

Na wykresach zaprezentowano liczbę aktywnych czujników w zależności od rodzaju protokołu, rozmiaru pakietu i okresów między pakietami. Sieć składała się z dwudziestu węzłów, które rozmieszczone zostały zgodnie z rozkładem jednorodnym.
Ogólnym trendem, który można zaobserwować w całej macierzy jest stopniowe skracanie się długości wykresu wraz ze wzrostem rozmiaru pakietu z danymi. Można również zauważyć, że wykresy, w których odstępy pomiędzy pakietami danych są dłuższe mają łagodniejszy i co za tym idzie dłuższy przebieg.
Dla rozmiaru pakietu równego 5B można dostrzec najdłuższy przebieg protokołu ALEACH. Jest to najbardziej widoczne w wykresach dla odstępów między pakietami równymi 15s i 20s. Pozostałe protokoły miały podobny przebieg. Najmniejsze różnice między ALEACH, a pozostałymi protokołami widoczne są na wykresie z czasem między pakietami równym 5s. Dla tej części macierzy najbardziej wyraźnie widać również wydłużający się przebieg wykresów wraz ze wzrostem przerwy między pakietami. 

\begin{figure}[H]
	\begin{center}
		\includegraphics[scale=0.5]{\ImgPath/charts/alive_nodes_uniform_20sensors_row1.png}
	\end{center}
	\caption{Aktywne węzły - liczba czujników: 20, rozmiar pakietu danych: 5B, rozkład jednorodny}
\end{figure}

Wykresy dla rozmiaru 50B mają na pierwszy rzut oka podobny przebieg, jak wykresy dla rozmiaru 5B, jednak widać kilka różnic między nimi. Linie poszczególnych protokołów nie nakładają się na siebie już w tak znaczący sposób, widać różnice pomiędzy poszczególnymi protokołami. Najlepsze wyniki, a więc najdłuższe działanie miał protokół ALEACH, który niezależnie od okresów między pakietami wykazał najłagodniejszy przebieg. Sytuacją, na którą warto zwrócić uwagę jest ta, która ma miejsce na wykresie dla czasu między pakietami równemu 15s. Tu zdecydowanie najkrótszy przebieg miał protokół LEACH.

\begin{figure}[H]
	\begin{center}
		\includegraphics[scale=0.5]{\ImgPath/charts/alive_nodes_uniform_20sensors_row2.png}
	\end{center}
	\caption{Aktywne węzły - liczba czujników: 20, rozmiar pakietu danych: 50B, rozkład jednorodny}
\end{figure}

 Podany wcześniej opis wykresów dla rozmiaru 50B mógłby być jednocześnie opisem tych dla rozmiaru 500B. Warto zwrócić uwagę na podobny przebieg protokołu LEACH na wykresie z okresami między pakietami równymi 15s. Był on krótszy od pozostałych protokołów. Różnica między protokołem ALEACH a pozostałymi protokołami była tym większa, im dłuższe były okresy między pakietami.

\begin{figure}[H]
	\begin{center}
		\includegraphics[scale=0.5]{\ImgPath/charts/alive_nodes_uniform_20sensors_row3.png}
	\end{center}
	\caption{Aktywne węzły - liczba czujników: 20, rozmiar pakietu danych: 500B, rozkład jednorodny}
\end{figure} 
  
Liczba aktywnych czujników malała najszybciej w wykresach o rozmiarze 5000B. Tu różnica między ALEACH, a pozostałymi protokołami była widoczna tylko na wykresie o okresie między pakietami równym 20s. W pozostałych protokoły miały podobny przebieg. 

\begin{figure}[H]
	\begin{center}
		\includegraphics[scale=0.5]{\ImgPath/charts/alive_nodes_uniform_20sensors_row4.png}
	\end{center}
	\caption{Aktywne węzły - liczba czujników: 20, rozmiar pakietu danych: 5000B, rozkład jednorodny}
\end{figure}


Na poniższej macierzy przedstawiono zestawienie wykresów obrazujących zależności między sumą energii wyrażonej w dżulach a czasem, rozmiarem pakietu oraz okresami między pakietami. Sieć składała się z dwudziestu węzłów, które rozmieszczone zostały zgodnie z rozkładem jednorodnym.
Na przedstawionych wykresach można zauważyć kilka wspólnych trendów. Pierwszym z nich jest inny od pozostałych przebieg protokołu SPIN, w przypadku którego energia jest szybciej wytracana. Warto zwrócić uwagę też na protokół ALEACH, którego wykres najczęściej był najdłuższy. Zależność ta jest najbardziej widoczna na wykresach o okresie między pakietami równym 15s i 20s. Ogólny przebieg wykresów protokołów (z wyjątkiem protokołu SPIN) był zbliżony. Można na wykresach zaobserwować trend wolniejszego wytracania energii wraz ze wzrostem okresów między pakietami.

Na wykresach dla rozmiaru pakietu 5B można zaobserwować największą różnicę pomiędzy protokołem SPIN a pozostałymi protokołami. Na uwagę zasługuje również zależność dla protokołu ALEACH, dla którego sieć tym wolniej wytraca energię im dłuższy jest okres między pakietami. Najbardziej zbliżony przebieg protokołów (z wyjątkiem protokołu SPIN) można zaobserwować na wykresie dla okresów między pakietami równemu 5s. Niewielkie różnice pojawiają się wraz ze wzrostem odstępu czasu pomiędzy pakietami danych.

\begin{figure}[H]
	\begin{center}
		\includegraphics[scale=0.5]{\ImgPath/charts/stored_energy_uniform_20sensors_row1.png}
	\end{center}
	\caption{Energia sieci - liczba czujników: 20, rozmiar pakietu danych: 5B, rozkład jednorodny}
\end{figure}
 
Wykresy o rozmiarach 50B i 500B mają podobny przebieg. Można tu zauważyć podobne trendy jak na wykresach o rozmiarze 5B. Warto zwrócić uwagę, że różnica między protokołem SPIN na wykresach o rozmiarze 50B jest relatywnie podobna, natomiast na wykresach o rozmiarze 500B rośnie wraz z wydłużeniem się odstępu między pakietami. Na wykresach o odstępach między pakietami danych równemu 15s protokół LEACH nieco szybciej wytraca energię w okolicach 180 sekundy działania sieci.

\begin{figure}[H]
	\begin{center}
		\includegraphics[scale=0.5]{\ImgPath/charts/stored_energy_uniform_20sensors_row2.png}
	\end{center}
	\caption{Energia sieci - liczba czujników: 20, rozmiar pakietu danych: 50B, rozkład jednorodny}
\end{figure}

\begin{figure}[H]
	\begin{center}
		\includegraphics[scale=0.5]{\ImgPath/charts/stored_energy_uniform_20sensors_row3.png}
	\end{center}
	\caption{Energia sieci - liczba czujników: 20, rozmiar pakietu danych: 500B, rozkład jednorodny}
\end{figure}
 
Wykresy pakietów o rozmiarze 5000B różnią się od pozostałych. Tu różnica między protokołem SPIN, a pozostałymi protokołami jest najmniejsza, a na wykresach o czasie między pakietami równym 5s i 10s jest niezauważalna. Warto zwrócić uwagę na przebieg protokołu ALEACH dla odstępów równych 20s. Tu jest on zdecydowanie dłuższy od pozostałych. 

\begin{figure}[H]
	\begin{center}
		\includegraphics[scale=0.5]{\ImgPath/charts/stored_energy_uniform_20sensors_row4.png}
	\end{center}
	\caption{Energia sieci - liczba czujników: 20, rozmiar pakietu danych: 5000B, rozkład jednorodny}
\end{figure}

Na poniższych wykresach przedstawiono zależność pomiędzy liczbą aktywnych czujników a czasem, rozmiarem pakietu i okresami między pakietami dla różnych protokołów. Sieć składała się z dwustu węzłów, które rozmieszczone zostały zgodnie z rozkładem jednorodnym.
Wykresy dla rozmiaru pakietu danych 5B są dość zróżnicowane. W przypadku okresów między pakietami równym 5s i 15s warto zwrócić uwagę na najszybsze zmniejszanie liczby aktywnych czujników przy zastosowaniu protokołu LEACH DCHS. Wykresem, który najwolniej opadał był w tych przypadkach ten, który ilustruje protokół LEACH. Podobne zróżnicowanie można zaobserwować przy wykresie dla okresów między pakietami równych 10s, jednak z tą różnicą, że punkt końcowy jest zbliżony dla wszystkich protokołów. Odmiennym wykresem od pozostałych jest ten dla odstępów między pakietami równych 20s. Tu liczba aktywnych czujników jest zbliżona we wszystkich protokołach.

\begin{figure}[H]
	\begin{center}
		\includegraphics[scale=0.5]{\ImgPath/charts/alive_nodes_uniform_200sensors_row1.png}
	\end{center}
	\caption{Aktywne węzły - liczba czujników: 200, rozmiar pakietu danych: 5B, rozkład jednorodny}
\end{figure}

Różnice w przebiegu wykresów ilustrujących ilość aktywnych czujników dla różnych protokołów można zauważyć również w przypadku pakietów o rozmiarze 50B. Warto zwrócić uwagę na protokół LEACH DCHS, który dla okresów między pakietami 5s, 10s, 15s przyjmuje wartości niższe, a w przypadku okresów równych 20s momentami wyższe od innych protokołów.  Z wyjątkiem protokołu LEACH DCHS wszystkie protokoły miały zbliżony przebieg. Dla okresów między pakietami równych 10s i 15s protokołem, po którego zastosowaniu czujniki pozostały najdłużej aktywne był ALEACH.

\begin{figure}[H]
	\begin{center}
		\includegraphics[scale=0.5]{\ImgPath/charts/alive_nodes_uniform_200sensors_row2.png}
	\end{center}
	\caption{Aktywne węzły - liczba czujników: 200, rozmiar pakietu danych: 50B, rozkład jednorodny}
\end{figure}

Wykresy pakietów o rozmiarze 500B są znacznie krótsze od opisywanych wcześniej, a więc liczba aktywnych czujników zmniejsza się szybciej. Warto zwrócić uwagę, że różnica między protokołami nie jest już zauważalna. Wykresy są dłuższe w miarę wydłużania się okresów między pakietami.

\begin{figure}[H]
	\begin{center}
		\includegraphics[scale=0.5]{\ImgPath/charts/alive_nodes_uniform_200sensors_row3.png}
	\end{center}
	\caption{Aktywne węzły - liczba czujników: 200, rozmiar pakietu danych: 500B, rozkład jednorodny}
\end{figure}

Najniżej znajdują się wykresy dla pakietów danych o rozmiarze 5000B. Wykresy te są najkrótsze w całej macierzy, a ten dla okresów między pakietami równych 5s jest krótszy niż 200 sekund. Pierwsze trzy wykresy posiadają zbliżony przebieg protokołów. Jedynie na wykresie dla czasu między kolejnymi pakietami z danymi równemu 20s można zaobserwować różnice między poszczególnymi protokołami, jednak zakończenie działania ostatniego czujnika jest zbliżone we wszystkich z nich.

\begin{figure}[H]
	\begin{center}
		\includegraphics[scale=0.5]{\ImgPath/charts/alive_nodes_uniform_200sensors_row4.png}
	\end{center}
	\caption{Aktywne węzły - liczba czujników: 200, rozmiar pakietu danych: 5000B, rozkład jednorodny}
\end{figure}

Na wykresach zobrazowano zależność pomiędzy sumą energii wyrażoną w dżulach a rozmiarem pakietu, okresami między pakietami dla różnych protokołów. Sieć składała się z dwustu węzłów, które rozmieszczone zostały zgodnie z rozkładem jednorodnym.

Wykresy dla pakietów o rozmiarze 5B charakteryzują się najwolniejszym opadaniem sumy energii sieci. Zależność ta wzrasta wraz z wydłużaniem się okresów między pakietami. Protokołem najszybciej wytracającym energię był SPIN. Wykresy obrazujące sumę energii przy zastosowaniu pozostałych protokołów posiadają podobny przebieg. Niewielkie różnice między protokołami można zaobserwować w przypadku okresów równych 15s. Tu najwolniej suma energii malała przy zastosowaniu protokołu LEACH.

\begin{figure}[H]
	\begin{center}
		\includegraphics[scale=0.5]{\ImgPath/charts/stored_energy_uniform_200sensors_row1.png}
	\end{center}
	\caption{Energia sieci - liczba czujników: 200, rozmiar pakietu danych: 5B, rozkład jednorodny}
\end{figure}

Przebieg podobnej długości do wykresów w których rozmiar pakietu danych wynosił 5B prezentują te o wielkości 50B. W ich przypadku można dostrzec dużą różnicę pomiędzy SPIN a pozostałymi protokołami.  Warto zwrócić uwagę na przebieg protokołu LEACH DCHS, w przypadku którego szybciej od pozostałych jest zużywana energia sieci. Różnica ta zmniejsza się wraz ze wzrostem okresów między pakietami. 

\begin{figure}[H]
	\begin{center}
		\includegraphics[scale=0.5]{\ImgPath/charts/stored_energy_uniform_200sensors_row2.png}
	\end{center}
	\caption{Energia sieci - liczba czujników: 200, rozmiar pakietu danych: 50B, rozkład jednorodny}
\end{figure}

Wykresy dla pakietów danych o rozmiarze 500B obrazują szybsze zużycie energii sieci od poprzednich.  Można zaobserwować zmniejszenie się różnicy między przebiegiem SPIN a pozostałymi protokołami. Widoczna jest również tendencja tym wolniejszego spadku sumy energii im dłuższy jest okres między pakietami. Protokoły z wyjątkiem SPIN  posiadały podobny przebieg.

\begin{figure}[H]
	\begin{center}
		\includegraphics[scale=0.5]{\ImgPath/charts/stored_energy_uniform_200sensors_row3.png}
	\end{center}
	\caption{Energia sieci - liczba czujników: 200, rozmiar pakietu danych: 500B, rozkład jednorodny}
\end{figure}

Wykresami prezentującymi najszybsze zużycie energii są te dla rozmiaru pakietu danych równemu 5000B. Najkrótszym wykresem jest ten dla okresów między pakietami o długości 5s, który jako jedyny w całej macierzy przedstawia czas działania sieci trwający mniej niż 200s. Warto zwrócić uwagę, że wszystkie protokoły, także SPIN, mają tu podobny przebieg. Różnica między SPIN a pozostałymi protokołami jest tym większa im dłuższy jest okres między pakietami.


\begin{figure}[H]
	\begin{center}
		\includegraphics[scale=0.5]{\ImgPath/charts/stored_energy_uniform_200sensors_row4.png}
	\end{center}
	\caption{Energia sieci - liczba czujników: 200, rozmiar pakietu danych: 5000B, rozkład jednorodny}
\end{figure}


\chapter{Wnioski}

Ogólna tendencja - im większy rozmiar pakietu oraz krótszy odstęp, tym  szybsze tempo wyłączania się węzłów sieci.

Algorytmy Flood oraz SPIN wykazują jednakowe tempo spadku liczby aktywnych czujników, niezależne od rozmiaru pakietu oraz okresu między pakietami. W przypadku również większość czujników wyłącza się w bardzo krótkim przedziale czasowym. Oznacza to, że zużycie energii przez węzły jest równomierne (wszystkie czujniki są przez cały czas swojej aktywności w trybie nasłuchiwania - w przeciwieństwie do rodziny alogrytmów typu LEACH, w których występują okresy ''uśpienia'' czujników).

Sieci korzystające z protokołów LEACH, ALEACH i LEACH DCHS wykazują wyraźnie dłuższe działanie niż sieci używające Flood i SPIN.
Różnice zacierają się dla większych rozmiarów pakietów i mniejszych odstępów czasu pomiędzy pakietami.

Różnice pomiędzy grupami protokołów są wyraźniejsze dla sieci z liczbą węzłów 200.



%-----------------
% Dodatki 
%-----------------
\appendix

\printbibliography

\zakonczenie  % wklejenie recenzji i opinii

\end{document}
%+++ END +++
